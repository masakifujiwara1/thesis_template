\chapter{結論}
\label{chap:end}
本研究では,経路追従行動をカメラ画像を用いた end-to-end 学習で模倣する岡田らの従来手法をベースに,データセットと学習器の入力へ目標方向を加えることで,経路選択をする機能の追加を提案した.また, シミュレータ上での実験を実環境に移す際に, 顕在化した課題について議論した. その結果, 2つのアプローチを試みることで学習時間を大幅に削減した. 加えて, 実環境での実験を行い, 有効性の検証を行った. 実験結果より, 学習器へ目標方向を与えることで, 指定した経路へ走行する挙動が確認できた.

%\input{experiments/preface}
%
% %!TEX root = ../thesis.tex

\section{}

%
