\chapter{結論}
\label{chap:end}
% 本研究では,経路追従行動をカメラ画像を用いた end-to-end 学習で模倣する岡田ら [2] の
% 従来手法をベースに,データセットと学習器の入力へ目標方向を加えることで,経路選択をす
% る機能の追加を提案した.またシミュレータ上で十字路,8 の字の環境を用いた実験を行い,
% 有効性の検証行った.実験結果より,学習器へ目標方向を与えることで,高い割合で指定した
% 経路へ走行する挙動が確認できた.

%\input{experiments/preface}
%
% %!TEX root = ../thesis.tex

\section{}

%
