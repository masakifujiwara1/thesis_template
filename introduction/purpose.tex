%!TEX root = ../thesis.tex

\section{目的}
% \ref{sec:relate-research}節で紹介した深層学習によるアプローチは,歩行者の複雑な動きを予測する上で有用である.しかし,環境への介入なしに予測結果を移動ロボットのナビゲーションへ応用している研究事例は少ない.また,歩行者の軌道予測を移動ロボットで活用するにあたっては,ロボットと歩行者間の相互作用も考慮すべきである.
% しかし,ロボットと人間を区別せずにエンコードして,より簡素なモデル構造
% そこで本論文では,環境への介入を要しないかたちでロボットの行動を考慮した歩行者の軌道予測手法を提案する.さらに,本手法を移動ロボットのナビゲーションに応用し,ロボットが安全かつ効率的に移動できるか検討する.

% ロボットの行動を考慮した歩行者の軌道予測を行い,その予測結果を用いたナビゲーションシステムを提案する.そして,本システムによるナビゲーション性能を実験を通じて検証し,その有効性を示すことを目指す.

本論文では,ロボットの行動を考慮した歩行者の軌道予測を可能にする深層学習モデルを提案する.さらに,提案モデルを移動ロボットのナビゲーションに応用し,実験によってナビゲーション性能を検証することで,その有効性を示すことを目指す.

