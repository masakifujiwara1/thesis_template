%!TEX root = ../thesis.tex

\section{背景}
近年,ロボット技術の進歩により,製造業,物流,サービス業など幅広い分野で自律移動ロボットの活用が進んでいる.人間とロボットが共存する環境において,ロボットは安全かつ効率的に動作するために,周囲の環境,特に人間の行動を理解し,それに適切に対応する必要がある.特に歩行者は動的な障害物であり,その予測困難な動きはロボットのナビゲーションにとって大きな課題となっている.
% 従来のロボットのナビゲーション手法は,歩行者の動きを単純なモデルで近似することが多く,複雑な環境や人間の行動の多様性に対応しきれていなかった.

この課題に対し,Social Force Model\cite{s-lstm}などの物理法則に基づくルールベース手法が提案されている.これらの手法は,歩行者の動きを物理的な力によってモデル化し,歩行者同士や障害物との相互作用を考慮することで,より現実的な動きの再現を目指している.しかし,これらの手法には限界があり,特に複雑な環境や多様な人間の行動を完全に表現することは困難である.

% Social Force Modelの他にIGPモデルもある.Robot nav-
% igation in dense human crowds: the case for cooperation. 中身を確認してから入れる

一方,近年では深層学習による歩行者の動きを予測する手法が注目されている.特に,リカレントニューラルネットワーク(RNN)\cite{rumelhart1986learning1,rumelhart1986learning2}とその発展形である長短期記憶(LSTM)\cite{hochreiter1997long}ネットワークを用いた手法が有望視されている.これらの手法は,過去の歩行者の動きの履歴を考慮し,未来の動きを予測することができる.また,グラフ畳み込みニューラルネットワーク(GCN)\cite{kipf2016semi-gcn}やグラフアテンションネットワーク(GAT)\cite{velickovic2017graph-gat}を用いた手法も注目を集めている.これらの手法は,人間をノード,人間同士の関係性(例えば,相対距離など)をエッジとしてグラフ化することで,相互作用を効率的に学習できる.

しかし,これらの手法を採用した研究において,移動ロボットを対象としたものが少なく,移動ロボットのナビゲーションに予測結果を応用した例の報告が少ない.さらに,移動ロボットを対象にしたものでも,歩行者へのセンサ取り付けなど環境への介入を必要とし,ロボットのナビゲーション性能を評価するための実験環境が整っていない場合が多い.

また,移動ロボットのナビゲーションに予測結果を応用する場合,歩行者同士の相互作用の他にロボットと人間との間にも相互作用が存在する.そのため,歩行者だけでなく,ロボットの行動も考慮した軌道予測が求められる.

本研究では,前述の2つの課題を克服するために,ロボットの行動を考慮した歩行者の軌道予測を行い,その予測結果を用いたナビゲーションシステムを提案する.そして,そのシステムによるナビゲーション性能を検証する.


\newpage
