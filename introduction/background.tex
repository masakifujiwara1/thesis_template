%!TEX root = ../thesis.tex

\section{背景}
近年,ロボット技術の進歩により,製造業,物流,サービス業などさまざまな分野で自律移動ロボットの活用が進んでいる.特に,人間とロボットが共存する環境では,ロボットは安全かつ効率的に動作するために,周囲の環境,特に人間の行動を理解し,それに適切に対応する必要がある.歩行者は動的な障害物であり,その予測不可能な動きはロボットのナビゲーションにとって大きな課題となっている.従来のロボットのナビゲーション手法は,歩行者の動きを単純なモデルで近似することが多く,複雑な環境や人間の行動の多様性に対応しきれていなかった.

\begin{figure}[hbtp]
  \centering
 \includegraphics[keepaspectratio, scale=0.8]
      {images/RaspberryPiMouse.png}
 \caption{Example}
 \label{Fig:Example}
\end{figure}

\subsubsection{etc...}
\newpage
