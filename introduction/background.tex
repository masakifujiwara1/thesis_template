%!TEX root = ../thesis.tex

\section{背景}
近年,ロボット技術の進歩に伴い,製造業,物流,サービス業など幅広い分野で自律移動ロボットの活用が進んでいる.人間とロボットが共存する環境では,ロボットが安全かつ効率的に動作するために,周囲の環境,特に人間の行動を正確に理解し,適切に対応することが求められる.なかでも歩行者は動的な障害物であり,その予測困難な動きはロボットのナビゲーションにとって大きな課題となっている.

この課題に対して,Social Force Model\cite{s-lstm}などの物理法則に基づくルールベース手法が提案されてきた.これらの手法では,歩行者の動きを物理的な力の相互作用によってモデル化し,歩行者同士や障害物との関係を考慮することで,より現実的な動きを再現できるように試みている.しかし,複雑な環境や多様な人間の行動を完全に表現することは依然として困難である.

一方,近年では深層学習による歩行者の動きを予測する手法が注目されている.特に,リカレントニューラルネットワーク(RNN)\cite{rumelhart1986learning1,rumelhart1986learning2}やその発展形である長短期記憶(LSTM)\cite{hochreiter1997long}ネットワークを用いたアプローチが有望視されている.これらの手法は,歩行者の過去の軌跡を考慮して未来の動きを予測できる.また,グラフ畳み込みニューラルネットワーク(GCN)\cite{kipf2016semi-gcn}やグラフアテンションネットワーク(GAT)\cite{velickovic2017graph-gat}を用いた手法も注目を集めている.これらの手法は,個々の歩行者をノード,相対距離などの関係性をエッジとみなしてグラフ化することで,歩行者間の相互作用を効率的に学習できる.

しかし,これらの深層学習ベースの手法を移動ロボットの文脈に直接適用した研究はまだ少ない.また,移動ロボットのナビゲーションに予測結果を応用した例の報告が少ない.さらに,移動ロボットを対象とする既存の研究でも,歩行者へのセンサ取り付けなど環境への介入が必要だったり,ロボットのナビゲーション性能を評価するための実験環境が十分に整備されていない場合が多い.

また,移動ロボットのナビゲーションに予測結果を応用する場合,歩行者同士の相互作用の他にロボットと人間との間にも相互作用が存在する.そのため,歩行者同士の相互作用だけでなく,ロボットと歩行者の相互作用も考慮する必要がある点も課題として挙げられる.つまり,ロボットの行動が歩行者の行動に影響を与えるため,歩行者のみならずロボットの行動も考慮した軌道予測が求められる.

そこで本研究では,上記2点の課題を克服するため,ロボットの行動を考慮した歩行者の軌道予測を行い,その予測結果を用いたナビゲーションシステムを提案する.そして,本システムによるナビゲーション性能を実験を通じて検証し,その有効性を示すことを目指す.

\newpage
