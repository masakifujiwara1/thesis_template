%!TEX root = ../thesis.tex

\section{グラフニューラルネットワーク}
グラフニューラルネットワーク(Graph Neural Network;GNN)は,ノード(頂点)とエッジ(辺)で構成されるグラフ構造のデータを処理するために設計されたニューラルネットワークの一種である.GNNでは,各ノードの特徴ベクトルを学習すると同時に,ノード間の関係性を考慮した上で予測や分類を行う.

GNNの基本的なアイデアは,各ノードの特徴ベクトルをその隣接ノードの特徴ベクトルと組み合わせて反復的に更新することである.この一連の更新は,メッセージパッシングと呼ばれ,複数の層を通じて繰り返し行われる.各層では,ノードの特徴ベクトルが隣接ノードからの情報を集約して更新されるため,最終的にグラフ全体にわたる依存関係を捉えることが可能となる.
代表的なGNNのアーキテクチャとしては,以下のようなものが挙げられる.
\begin{itemize}
  \item \textbf{Graph Convolutional Networks (GCN)}\\
  グラフ畳み込みネットワークは,畳み込みに相当する操作をグラフに適用し,隣接ノードの情報を考慮して各ノードの特徴を更新する.
  \item \textbf{Graph Attention Networks (GAT)}\\
  グラフアテンションネットワークは,隣接するノード間に異なる重みを付与することで,重要度の高いノードからの情報を強調する.アテンション機構を取り入れることで,より柔軟な学習が可能となる.
  \item \textbf{GraphSAGE}\\
  ノードサンプリングと情報集約の手法を組み合わせることで,大規模なグラフを効率的に扱う.各ステップでサブグラフを取り出し,必要最小限の隣接ノードのみを考慮して学習を行うため,計算コストを抑制できる.
\end{itemize}

GNNは,ソーシャルネットワーク分析,化学分子の特性予測,知識グラフの補完など,さまざまな分野で応用されている.グラフ構造データの持つ複雑なパターンを学習し,従来の手法では困難とされてきた問題を解決可能である点がGNNの大きな利点である.

\begin{figure}[H]
  \centering
 \includegraphics[keepaspectratio, scale=0.4]
      {images/RaspberryPiMouse.png}
 \caption{Neural Network}
 \label{Fig:hoge3}
\end{figure}   

\newpage
