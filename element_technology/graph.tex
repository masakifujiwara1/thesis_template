%!TEX root = ../thesis.tex

\section{グラフニューラルネットワーク}
グラフニューラルネットワーク(GNN)は,グラフ構造データを処理するために設計されたニューラルネットワークの一種である.GNNは,ノード(頂点)とエッジ(辺)からなるグラフを入力として受け取り,各ノードの特徴を学習し,ノード間の関係性を考慮した上で予測や分類を行う.

GNNの基本的なアイデアは,各ノードの特徴ベクトルをその隣接ノードの特徴ベクトルと組み合わせて更新することである.このプロセスは,メッセージパッシングと呼ばれ,複数の層を通じて繰り返される.各層では,ノードの特徴ベクトルが隣接ノードからの情報を集約し,更新される.

代表的なGNNのアーキテクチャには,以下のようなものがある:
\begin{itemize}
  \item \textbf{Graph Convolutional Networks (GCN)}: グラフ畳み込みネットワークは,畳み込み操作をグラフデータに適用することで,ノードの特徴を更新する.
  \item \textbf{Graph Attention Networks (GAT)}: グラフアテンションネットワークは,各エッジに異なる重みを付けることで,重要な隣接ノードからの情報を強調する.
  \item \textbf{GraphSAGE}: サンプリングと集約の手法を用いて,効率的に大規模なグラフデータを処理する.
\end{itemize}

GNNは,ソーシャルネットワーク分析,化学分子の特性予測,知識グラフの補完など,さまざまな分野で応用されている.これらのネットワークは,グラフ構造データの複雑なパターンを捉える能力があり,従来の手法では難しかった問題を解決することができる.

\begin{figure}[hbtp]
  \centering
 \includegraphics[keepaspectratio, scale=0.5]
      {images/RaspberryPiMouse.png}
 \caption{Neural Network}
 \label{Fig:hoge3}
\end{figure}   

\newpage
