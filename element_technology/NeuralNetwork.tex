%!TEX root = ../thesis.tex

\section{ニューラルネットワーク}
ニューラルネットワーク(Neural Network)は,人間の脳の神経回路に着想を得た計算モデルであり,機械学習や人工知能の分野で重要な役割を果たしている.大量のデータから学習し,パターン認識や予測などの複雑なタスクを遂行する能力を持つ.パーセプトロンを入れる

ニューラルネットワークは,多数のノード(ニューロン)が層状に接続された構造を持つ.各ノードは,入力信号を受け取り,重み付けを行い,活性化関数を適用することで出力信号を生成する.これらのノードは,入力層,隠れ層,出力層の3つの層に分けられる.

\begin{itemize}
     \item 入力層\\
     入力データを受け取る層であり,各ノードは入力データの各特徴に対応する.
     \item 隠れ層\\
     入力層と出力層の間にある層であり,複数の隠れ層を持つことができ,複雑な非線形関係を学習することが可能になる.
     \item 出力層\\
     ネットワークの最終的な出力を生成する層である.
\end{itemize}

各層のノードは,次の層のノードと接続されており,接続には重みが割り当てられている.学習プロセスでは,これらの重みを調整することで,ネットワークが目的のタスクを実行できるように最適化される.
% この最適化は,一般的に損失関数を最小化するように,勾配降下法などの最適化アルゴリズムを用いて行われる.
多層構造を持つことで,複雑なパターンや特徴を学習し,従来では困難であった高精度な予測を実現した学習手法を深層学習と呼ぶ.深層学習は,画像認識,自然言語処理,音声認識といった多岐にわたる分野で活用されており,YOLO\cite{redmon2016you-yolo}やChatGPT\cite{radford2018improving-gpt, radford2019language-gpt,brown2020language-gpt}などの成功例が多数存在し,高い性能が評価されている.

\begin{figure}[hbtp]
     \centering
    \includegraphics[keepaspectratio, scale=0.5]
         {images/RaspberryPiMouse.png}
    \caption{Neural Network}
    \label{Fig:MLP}
\end{figure}   

\newpage
