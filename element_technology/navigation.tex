%!TEX root = ../thesis.tex

\section{ナビゲーション}\label{sec:navigation-stack}
LiDARやオドメトリなどのセンサ情報を利用して自律走行する移動ロボットの多くは,ROS Navigation stack\cite{nav1,nav2}を採用している.例えば,つくば市内の遊歩道を移動ロボットが自律走行する技術チャレンジ「つくばチャレンジ」において,原らの技術調査\cite{robomech2024-hara}によると,少なくとも参加76チーム中22チームが自律走行でROS Navigation stackを使用していたと報告されている.これは,オープンソースのソフトウェアとしては最多の利用数であった.ROS Navigation stackには,以下の主要な機能が含まれている.

\begin{itemize}
     \item \textbf{自己位置推定(Localization)}\\
     AMCL(Adaptive Monte Carlo Localization)アルゴリズムを用いて,ロボットの現在位置を推定する.
     \item \textbf{地図生成(Mapping)}\\
     SLAM(Simultaneous Localization and Mapping)技術を活用し,環境のマッピングを行う.
     \item \textbf{経路計画(Path Planning)}\\
     Dijkstra法\cite{dijkstra2022note}やA*アルゴリズム\cite{hart1968formal-astar}を用いて,最適経路を計画する.
     \item \textbf{障害物回避(Obstacle Avoidance)}\\
     センサデータに基づいて障害物を検知し,動的に回避しながら移動する.
\end{itemize}

\newpage
