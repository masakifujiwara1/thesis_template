\documentclass[uplatex, a4paper, 12pt, openany, oneside]{jsbook}

\usepackage[dvipdfmx]{graphicx}
\usepackage[dvipdfmx]{color}
\usepackage[dvipdfmx, bookmarks=true, setpagesize=false]{hyperref}
\usepackage{pxjahyper}

\usepackage{thesis}
\usepackage{here}
\usepackage{url}

\usepackage[hang,small,bf]{caption}
\usepackage[subrefformat=parens]{subcaption}
\captionsetup{compatibility=false}

\thesis{卒 業 論 文}
\title{
  \centering
    \scalebox{1.0}{視覚と行動のend-to-end学習により}
    \vspace{-0.3zh}
    \scalebox{1.0}{経路追従行動をオンラインで模倣する手法の提案}
    \vspace{-0.3zh}
    \scalebox{1.0}{(目標方向による経路選択機能の追加と検証)}
    % \vspace{-0.1zh}
    \vspace{0.5cm}
    \scalebox{0.6}{A proposal for an online imitation method of path-tracking}\\
    \vspace{-0.6zh}
    \scalebox{0.6}{behavior by end-to-end learning of vision and action}\\
    \vspace{-0.6zh}
    \scalebox{0.6}{(Addition of path selection function and verification by target direction)}\\
    \vspace{-0.6zh}
}
\setlength{\textwidth}{\fullwidth}
\setlength{\evensidemargin}{\oddsidemargin}

\date{\today}
\vspace{-15.0zh}
\teacher{林原 靖男 教授}
\vspace{-15.0zh}
\organization{千葉工業大学 先進工学部 未来ロボティクス学科}
\author{19C1101 藤原 柾}
\vspace{-15zh}

\renewcommand{\baselinestretch}{1.2}
\begin{document}

%% Front Matter
\frontmatter{}
%
\chapter{提案手法}
\label{chap:suggest}
%
本章では, 従来手法をベースとする提案手法についての概要, 提案手法における学習フェーズ, テストフェーズ, 目標方向, ネットワーク構造についての5節に分けて述べる.
%\input{introduction/preface}
%
%!TEX root = ../thesis.tex

\section{提案手法の概要}
従来手法で用いていたデータセットと学習器の入力へ, 「直進」「左折」などの目標方向を追加する. これにより, 学習器の出力による自律移動において, 経路を選択する機能の追加を行った. なお, 追加した要素以外は従来手法と同様である.

% \subsection{RoboCup}

% \begin{figure}[hbtp]
%   \centering
%  \includegraphics[keepaspectratio, scale=0.4]
%       {images/deeplearning_model.png}
%  \caption{Neural network}
%  \label{Fig:Neural network}
% \end{figure}

% \subsubsection{etc...}
\newpage

%!TEX root = ../thesis.tex

\section{学習フェーズ}
提案手法で用いる学習フェーズのシステムを\figref{Fig:suggest_learning_sys}に示す. 自律移動を行う地図を用いたルールベース制御器から目標方向を生成し, データセットに加えている. 
なお, 厳密にはルールベース制御を構成するwaypoint\_navにより目標方向を生成している. 
提案手法では, 
% \figref{Fig:overview}に示すように
LiDARとオドメトリを入力とする地図を用いたルールベース制御器による自律移動を, カメラ画像と目標方向を用いて模倣学習する.

% \vspace{3cm}
\begin{figure}[hbtp]
     \centering
    \includegraphics[keepaspectratio, scale=0.45]
         {images/conventional.png}
    \caption{Learning phase system of conventional method}
    \label{Fig:conventional}
   \end{figure}   

\begin{figure}[hbtp]
  \centering
 \includegraphics[keepaspectratio, scale=0.45]
      {images/suggest_learning_sys2.png}
 \caption{Learning phase system of proposed method}
 \label{Fig:suggest_learning_sys}
\end{figure}

% \begin{figure}[hbtp]
%   \centering
%  \includegraphics[keepaspectratio, scale=0.6]
%       {images/overview.png}
%  \caption{Overview learning phase}
%  \label{Fig:overview}
% \end{figure}

% \subsubsection{etc...}
\newpage

%!TEX root = ../thesis.tex

\section{テストフェーズ}
提案手法におけるテストフェーズでは\figref{Fig:suggest_test_phase}で示すように, 学習器の入力へ目標方向を加えた. なお, テストフェーズも学習フェーズと同様に, 地図を用いたルールベース制御器から目標方向を生成している. 本来ならば, 目的地までカメラ画像のみで自律移動するためには, 目標方向を画像から自動的に作成する仕組みが必要となる. 
% figに動作の様子を示す. 
カメラ画像と目標方向を入力した学習器の出力による自律走行時, 目標方向によって任意の経路を選択する.

\vspace{3cm}

\begin{figure}[hbtp]
  \centering
 \includegraphics[keepaspectratio, scale=0.45]
      {images/suggest_test_phase.png}
 \caption{Learning phase system of proposed method}
 \label{Fig:suggest_test_phase}
\end{figure}

% \subsubsection{etc...}
\newpage

%!TEX root = ../thesis.tex

\section{目標方向}
本研究で用いた目標方向と, そのデータ形式である目標方向指令について述べる. 目標方向を\figref{Fig:direction}に示す. 経路と分岐路において「道なり」に走行(Go straight), 分岐路において「直進(Go straight)」, 「左折(Turn left)」, 「右折(Turn right)」の3つとする.

\begin{figure}[hbtp]
  \centering
 \includegraphics[keepaspectratio, scale=0.45]
      {images/direction.png}
 \caption{Target direction}
 \label{Fig:direction}
\end{figure}

学習器には, 上記の3つの目標方向を要素数3, 次元数1のint型の配列で表現した”目標方向指令”を入力する. 目標方向指令のデータ形式を\tabref{table:direction}に示す.

\begin{table}[hbtp]
  \caption{Target direction list}
  \label{table:direction}
  \centering
  \begin{tabular}{|c|c|c|c|}
    \hline
    Target Direction  & Go straight & Turn left & Trun right\\
    \hline
    Data & [100, 0, 0] & [0, 100, 0] & [0, 0, 100]\\
    \hline
  \end{tabular}
\end{table}

% \begin{figure}[hbtp]
%   \centering
%  \includegraphics[keepaspectratio, scale=0.45]
%       {images/direction.png}
%  \caption{Learning phase system of proposed method}
%  \label{Fig:direction}
% \end{figure}

% \subsubsection{etc...}
\newpage

%!TEX root = ../thesis.tex

\section{ネットワーク構造}
提案手法で用いた学習器のネットワークを\figref{Fig:network_structure}に示す. また, ハイパーパラメータについて\tabref{table:param1}に示す. 64x48のRGB画像を入力とする入力層1層, 畳み込み層3層, 全結合層2層を持つ6層のCNNと, このCNNの出力と目標方向指令を入力する入力層1層, 全結合層2層, 出力層1層の全10層から構成されている. 出力はヨー方向の角速度である.

\begin{figure}[hbtp]
  \centering
 \includegraphics[keepaspectratio, scale=0.43]
      {images/network_structure2.png}
 \caption{Structure of network}
 \label{Fig:network_structure}
\end{figure}

\begin{table}[hbtp]
  \caption{Parameters of network}
  \label{table:param1}
  \centering
  \begin{tabular}{|c|c|}
    \hline
    Input data & Image(64x48 pixels, RGB channels), Target direction \\
    \hline
    Optimizer & Adam($alpha = 0.001, beta1 = 0.9, beta2 =  0.999, eps = 1e^{-2}$)\\
    \hline
    Loss function & Softmax-cross-entropy\\
    \hline
    Output data & Angular velocity\\
    \hline
  \end{tabular}
\end{table}

% \begin{figure}[hbtp]
%   \centering
%  \includegraphics[keepaspectratio, scale=0.45]
%       {images/network_structure.png}
%  \caption{Learning phase system of proposed method}
%  \label{Fig:network_structure}
% \end{figure}

% \subsubsection{etc...}
\newpage

%

%
%% Main Matter
\mainmatter{}
%
\chapter{序論}
\label{chap:introduction}
%
%\input{introduction/preface}
%
%!TEX root = ../thesis.tex

\section{背景}
近年,ロボット技術の進歩に伴い,製造業,物流,サービス業など幅広い分野で自律移動ロボットの活用が進んでいる.人間とロボットが共存する環境では,ロボットが安全かつ効率的に動作するために,周囲の環境,特に人間の行動を正確に理解し,適切に対応することが求められる.なかでも歩行者は動的な障害物であり,その予測困難な動きはロボットのナビゲーションにとって大きな課題となっている.

この課題に対して,Social Force Model\cite{s-lstm}などの物理法則に基づくルールベース手法が提案されてきた.これらの手法では,歩行者の動きを物理的な力の相互作用によってモデル化し,歩行者同士や障害物との関係を考慮することで,より現実的な動きを再現できるように試みている.しかし,複雑な環境や多様な人間の行動を完全に表現することは依然として困難である.

一方,近年では深層学習による歩行者の動きを予測する手法が注目されている.特に,リカレントニューラルネットワーク(RNN)\cite{rumelhart1986learning1,rumelhart1986learning2}やその発展形である長短期記憶(LSTM)\cite{hochreiter1997long}ネットワークを用いたアプローチが有望視されている.これらの手法は,歩行者の過去の軌跡を考慮して未来の動きを予測できる.また,グラフ畳み込みニューラルネットワーク(GCN)\cite{kipf2016semi-gcn}やグラフアテンションネットワーク(GAT)\cite{velickovic2017graph-gat}を用いた手法も注目を集めている.これらの手法は,個々の歩行者をノード,相対距離などの関係性をエッジとみなしてグラフ化することで,歩行者間の相互作用を効率的に学習できる.

しかし,これらの深層学習ベースの手法を移動ロボットの文脈に直接適用した研究はまだ少ない.また,移動ロボットのナビゲーションに予測結果を応用した例の報告が少ない.さらに,移動ロボットを対象とする既存の研究でも,歩行者へのセンサ取り付けなど環境への介入が必要だったり,ロボットのナビゲーション性能を評価するための実験環境が十分に整備されていない場合が多い.

また,移動ロボットのナビゲーションに予測結果を応用する場合,歩行者同士の相互作用の他にロボットと人間との間にも相互作用が存在する.そのため,歩行者同士の相互作用だけでなく,ロボットと歩行者の相互作用も考慮する必要がある点も課題として挙げられる.つまり,ロボットの行動が歩行者の行動に影響を与えるため,歩行者のみならずロボットの行動も考慮した軌道予測が求められる.

そこで本研究では,上記2点の課題を克服するため,ロボットの行動を考慮した歩行者の軌道予測を行い,その予測結果を用いたナビゲーションシステムを提案する.そして,本システムによるナビゲーション性能を実験を通じて検証し,その有効性を示すことを目指す.

\newpage

%!TEX root = ../thesis.tex

\section{関連研究}
歩行者の複雑な動きを予測するために,深層学習を応用しようという研究が,近年注目を集めている.

\subsection{LSTMによる歩行者の軌道予測の応用事例}
Alexandreらは,人間の動きを学習し,未来の軌跡を予測できるLSTMモデルを提案している.この研究では,

\subsection{GNNによる歩行者の軌道予測の応用事例}
hoge

\subsection{ロボットの将来の行動を軌道予測に用いた事例}
丹野らは,歩行者の軌道予測に過去の動きだけでなく,ロボットが選択する将来の動きを考慮して予測をしている.実験では,ロボットの将来の行動を考慮しない場合と比較して,ロボットの動きに合わせて変化する歩行者の動きを予測できている.

\subsubsection{etc...}
\newpage

%!TEX root = ../thesis.tex

\section{目的}
% \ref{sec:relate-research}節で紹介した深層学習によるアプローチは,歩行者の複雑な動きを予測する上で有用である.しかし,環境への介入なしに予測結果を移動ロボットのナビゲーションへ応用している研究事例は少ない.また,歩行者の軌道予測を移動ロボットで活用するにあたっては,ロボットと歩行者間の相互作用も考慮すべきである.
% しかし,ロボットと人間を区別せずにエンコードして,より簡素なモデル構造
% そこで本論文では,環境への介入を要しないかたちでロボットの行動を考慮した歩行者の軌道予測手法を提案する.さらに,本手法を移動ロボットのナビゲーションに応用し,ロボットが安全かつ効率的に移動できるか検討する.

% ロボットの行動を考慮した歩行者の軌道予測を行い,その予測結果を用いたナビゲーションシステムを提案する.そして,本システムによるナビゲーション性能を実験を通じて検証し,その有効性を示すことを目指す.

本論文では,ロボットの行動を考慮した歩行者の軌道予測を可能にする深層学習モデルを提案する.さらに,提案モデルを移動ロボットのナビゲーションに応用し,実験によってナビゲーション性能を検証することで,その有効性を示すことを目指す.


%!TEX root = ../thesis.tex

\section{本論文の構成}
本論文は,全7章で構成されている.第1章では,研究の背景や関連研究,目的を述べた.第2章では,本研究に用いた要素技術について述べる.第3章では,本研究で提案する手法について具体的に説明し,設計したネットワークの基本性能を確認した.第4章では,提案手法の有効性を検証する実験について述べる.第5章では,予測結果を用いたナビゲーションシステムの構築について述べる.第6章では,実験で予測結果を利用したナビゲーションの有効性を検証する.第7章では,本研究の成果をまとめ,今後の展望について述べる.

\newpage

%

\chapter{要素技術}
\label{chap:technology}
%
本章では, 本研究で用いた深層学習に関連した要素技術と, ベースとなる従来手法にていて述べる.
%\input{introduction/preface}
%
%!TEX root = ../thesis.tex

\section{Deep learning}
Deep learningは, 画像や音声などのデータに特に適しており, 近年では
自然言語処理や医療画像解析などさまざまな分野で活用されている.
人間の脳のような深い層の構造を持つ人工ニューラルネットワークに基づく機械学習手法である. 
人工ニューラルネットワークは, 入力データから出力データを予測するために, 多数のニューロンを
用いて情報を処理する. この人工ニューラルネットワークを多層構造にすることで, より深い情報
処理を行うことができる. これにより, 高度な識別や分類タスクなどを行うことを可能にしている. 
一般的な構造を\ref{Fig:Neural network} に示す.

% \subsection{RoboCup}

\begin{figure}[hbtp]
  \centering
 \includegraphics[keepaspectratio, scale=0.4]
      {images/deeplearning_model.png}
 \caption{Neural network}
 \label{Fig:Neural network}
\end{figure}

% \subsubsection{etc...}
\newpage

%!TEX root = ../thesis.tex

\section{end-to-end学習}
end-to-end学習とは, 人工ニューラルネットワークを使用して, 入力データから出力を直接
生成する方法のことを指す. \par 実世界における自動運転を例に挙げる. end-to-end学習を用いない場合, 人物や障害物などの物体認識, 車線の検出, 経路計画, ステアリングの制御など, 多くのタスクを解決する必要が
ある. しかし, end-to-endを用いることで, 先程のタスクを解決することなく, 車両が撮影したカメラ映像から直接, 運転操作を行うことができる. 
% \subsection{RoboCup}
\vspace{5cm}

\begin{figure}[hbtp]
  \centering
 \includegraphics[keepaspectratio, scale=0.7]
      {images/end-to-end.png}
 \caption{Structure of end-to-end learning}
 \label{Fig:end-to-end}
\end{figure}

% \subsubsection{etc...}
\newpage

%!TEX root = ../thesis.tex

\section{Convolution Neural Network}
畳み込みニューラルネットワーク(convolutional neural network:CNN)は人工ニューラルネットワークのモデルの一種である. このモデルは, 画像や音声などの多次元の配列で表される複雑なデータを処理するために特別に設計されている. CNNは次のような特徴を持つ層で構成されている.
\begin{enumerate}
  \item 畳み込み層\\入力データをフィルタ(カーネル)を用いて特徴を抽出する.
  \item プーリング層\\特徴を残しつつ, 畳み込み層の出力を圧縮する. これにより, 画像であればピクセル数が減少し, 計算量が大幅に減らすことができる.
  \item 全結合層\\畳み込み層とプーリング層の出力をまとめて処理する.
\end{enumerate}

Krizhevskyら\cite{AlexNet}は\figref{Fig:AlexNet}で示すような, 8つの層からなるネットワークを用いて, 15万の高解像度画像を1,000の異なるクラスに分類するタスクをエラー率15.3\%で達成し, 画像分類コンペティションである ILSVRC(ImageNet Large Scale VisualRecognition Competition)2012で優勝をおさめた.
\begin{figure}[hbtp]
  \centering
 \includegraphics[keepaspectratio, scale=0.55]
      {images/AlexNet.png}
 \caption{AlexNet from \cite{AlexNet}}
 \label{Fig:AlexNet}
\end{figure}
\newpage
\par SimonyanらはCNNの層の深さが精度に与える影響を調査した. 最大19層の深い畳み込みネットワークを評価した結果, モデルを深層にすることが分類精度に有利であることが示された.
ILSVRC2012の優勝モデルであるAlexNetは8層, ILSVRC2013で提案されたZFNetは同様の8層であることから, 当時のCNNとしては圧倒的に深い層を持つモデルであった. このような深い畳み込みネットワークは, 深層学習における重要な発展の一つとされている.
\begin{figure}[hbtp]
  \centering
 \includegraphics[keepaspectratio, scale=0.5]
      {images/VGG.png}
 \caption{VGG from \cite{AlexNet}}
 \label{Fig:VGG}
\end{figure}


% \vspace{5cm}

% \begin{figure}[hbtp]
%   \centering
%  \includegraphics[keepaspectratio, scale=0.7]
%       {images/end-to-end.png}
%  \caption{Structure of end-to-end learning}
%  \label{Fig:end-to-end}
% \end{figure}

% \subsubsection{etc...}
\newpage

%!TEX root = ../thesis.tex

\section{地図を用いたルールベースの制御器}
従来手法と提案手法において, 教師信号として用いる地図を用いたルールベース制御器について述べる. 地図を用いたルールベース制御器は, ROS Navigation\_stack\cite{navigation:online}へ目標位置(waypoint)の指示を行うwaypoint\_nav\cite{waypoint_nav:online}を組み合わせたものである. なお, 後述するが提案手法ではwaypoint\_navの役割が増えている. ROS Navigation\_stackでは以下のような処理が行われる. 

\begin{itemize}
  \item ロボットの現在位置を推定する
  % \item 移動目標地点を決定する
  \item 移動目標地点までの経路を決定する
  \item 経路にしたがった行動をロボットに指示する
\end{itemize}

また, \figref{Fig:navigation}に示すように, 事前に作成した占有格子地図と測域センサ, オドメトリを用いて, 地図上での自己位置をParticle\_Filterを用いて近似することで推定する「amcl」. 障害物認識などによる局所的, または地図全体の大域的なコスト計算, その結果に基づいた経路計画, それに従ったモータ指令を行う「move\_base」などのパッケージによって構成されている自律移動を行うためのメタパッケージである. 従来手法, 提案手法ともにモータ指令を並進速度と角速度にわけた. なお, 並進速度は一定とした.

% \vspace{5cm}

\begin{figure}[hbtp]
  \centering
 \includegraphics[keepaspectratio, scale=0.7]
      {images/rule-based.png}
 \caption{A rule-based controller using a map}
 \label{Fig:navigation}
\end{figure}

% \subsubsection{etc...}
\newpage
%!TEX root = ../thesis.tex

\section{従来手法}
本研究のベースとなる岡田らの研究について述べる. 先に述べたように, 本論文では岡田らの手法を「従来手法」と呼ぶ. 従来手法は, 地図を用いたルールベース制御器による走行を模倣学習し, 似た行動を画像を用いて行う手法である.\par
\figref{Fig:imitation_sys}に, 経路追従行動を視覚に基づいてオンラインで模倣するシステムを示す. 手法は模倣学習により, 学習器の訓練をする学習フェーズと訓練した結果を検証するテストフェーズに分かれる. なお, 両フェーズで用いる並進速度は一定の値を用いる.

\vspace{3cm}

\begin{figure}[hbtp]
  \centering
 \includegraphics[keepaspectratio, scale=0.5]
      {images/imitation_sys.png}
 \caption{Learning phase of conventional method from \cite{mech}}
 \label{Fig:imitation_sys}
\end{figure}

\newpage

\subsection{学習フェーズ}
学習フェーズは, 模倣学習によって学習器の訓練を行うフェーズである. 測域センサとオドメトリを入力とする地図を用いたルールベース制御器で自律移動する. 具体的には, ROS navigation\_stackパッケージを利用して, ロボットに自律移動させる. 学習フェーズでは, ロボットの中央, 左, 右に傾けて取り付けた3つのカメラを用いて画像を取得する. 
\par
自律移動させる際に, 取得するデータ量を増加させること, 及び過学習の抑制を目的として, \tabref{table:angular}に示すような処理を行う. また, 地図を用いたルールベース制御器による走行をそのまま模倣学習するのではなく, 少し蛇行するように自律移動させることで, 経路に戻るような挙動も学習できるようになっている. \figref{Fig:dakou}に示すように, 実際にロボットを制御する行動と経路に従う行動を別に扱うことで, 常に経路に従う行動をデータセットに加えることを可能にしている. 
% ロボットに搭載したカメラ画像と地図を用いたルールベース制御器が出力するロボットのヨー方向の角速度の値をデータセットに加える. 
% このデータセットを用いて, リアルタイムに模倣学習を行う. 
\par
岡田らの手法では, データセットの収集方法にいくつかの種類がある. 
% 本論文では前報\cite{okada2}において最も経路追従の成功率が高い手法を用いて, ロボットに模倣学習をさせる.
本論文では, その中で最も経路追従の成功率が高い手法を用いて, ロボットに模倣学習をさせる.

\begin{table}[hbtp]
  \caption{Angular velocity offset}
  \label{table:angular}
  \centering
  \begin{tabular}{|c|c|}
    \hline
    Left camera  & Angular velocity of a rule-based controller using a map + 0.2 rad/s\\
    \hline
    Center camera  & Angular velocity of a rule-based controller using a map  \\
    \hline
    Right camera  & Angular velocity of a rule-based controller using a map - 0.2 rad/s   \\
    \hline
  \end{tabular}
\end{table}

\begin{figure}[hbtp]
  \centering
 \includegraphics[keepaspectratio, scale=0.5]
      {images/3action4.png}
 \caption{Output of rule-based controller using a map and actual robot behavior}
 \label{Fig:dakou}
\end{figure}

\subsection{テストフェーズ}
学習器の訓練後, \figref{Fig:test_phase}で示すテストフェーズへ移行する. このフェーズでは, 学習器にカメラ画像を入力し, 出力されるヨー方向の角速度を用いて自律移動することで, 訓練後の学習結果を評価する. なお, テストフェーズでは中央のカメラのみを用いる.

\vspace{1cm}


\begin{figure}[hbtp]
  \centering
 \includegraphics[keepaspectratio, scale=0.40]
      {images/test_phase3.png}
 \caption{Test phase of conventional method}
 \label{Fig:test_phase}
\end{figure}

% \vspace{3cm}

% \subsubsection{etc...}
\newpage
%

\chapter{提案手法}
\label{chap:suggest}
%
本章では, 従来手法をベースとする提案手法についての概要, 提案手法における学習フェーズ, テストフェーズ, 目標方向, ネットワーク構造についての5節に分けて述べる.
%\input{introduction/preface}
%
%!TEX root = ../thesis.tex

\section{提案手法の概要}
従来手法で用いていたデータセットと学習器の入力へ, 「直進」「左折」などの目標方向を追加する. これにより, 学習器の出力による自律移動において, 経路を選択する機能の追加を行った. なお, 追加した要素以外は従来手法と同様である.

% \subsection{RoboCup}

% \begin{figure}[hbtp]
%   \centering
%  \includegraphics[keepaspectratio, scale=0.4]
%       {images/deeplearning_model.png}
%  \caption{Neural network}
%  \label{Fig:Neural network}
% \end{figure}

% \subsubsection{etc...}
\newpage

%!TEX root = ../thesis.tex

\section{学習フェーズ}
提案手法で用いる学習フェーズのシステムを\figref{Fig:suggest_learning_sys}に示す. 自律移動を行う地図を用いたルールベース制御器から目標方向を生成し, データセットに加えている. 
なお, 厳密にはルールベース制御を構成するwaypoint\_navにより目標方向を生成している. 
提案手法では, 
% \figref{Fig:overview}に示すように
LiDARとオドメトリを入力とする地図を用いたルールベース制御器による自律移動を, カメラ画像と目標方向を用いて模倣学習する.

% \vspace{3cm}
\begin{figure}[hbtp]
     \centering
    \includegraphics[keepaspectratio, scale=0.45]
         {images/conventional.png}
    \caption{Learning phase system of conventional method}
    \label{Fig:conventional}
   \end{figure}   

\begin{figure}[hbtp]
  \centering
 \includegraphics[keepaspectratio, scale=0.45]
      {images/suggest_learning_sys2.png}
 \caption{Learning phase system of proposed method}
 \label{Fig:suggest_learning_sys}
\end{figure}

% \begin{figure}[hbtp]
%   \centering
%  \includegraphics[keepaspectratio, scale=0.6]
%       {images/overview.png}
%  \caption{Overview learning phase}
%  \label{Fig:overview}
% \end{figure}

% \subsubsection{etc...}
\newpage

%!TEX root = ../thesis.tex

\section{テストフェーズ}
提案手法におけるテストフェーズでは\figref{Fig:suggest_test_phase}で示すように, 学習器の入力へ目標方向を加えた. なお, テストフェーズも学習フェーズと同様に, 地図を用いたルールベース制御器から目標方向を生成している. 本来ならば, 目的地までカメラ画像のみで自律移動するためには, 目標方向を画像から自動的に作成する仕組みが必要となる. 
% figに動作の様子を示す. 
カメラ画像と目標方向を入力した学習器の出力による自律走行時, 目標方向によって任意の経路を選択する.

\vspace{3cm}

\begin{figure}[hbtp]
  \centering
 \includegraphics[keepaspectratio, scale=0.45]
      {images/suggest_test_phase.png}
 \caption{Learning phase system of proposed method}
 \label{Fig:suggest_test_phase}
\end{figure}

% \subsubsection{etc...}
\newpage

%!TEX root = ../thesis.tex

\section{目標方向}
本研究で用いた目標方向と, そのデータ形式である目標方向指令について述べる. 目標方向を\figref{Fig:direction}に示す. 経路と分岐路において「道なり」に走行(Go straight), 分岐路において「直進(Go straight)」, 「左折(Turn left)」, 「右折(Turn right)」の3つとする.

\begin{figure}[hbtp]
  \centering
 \includegraphics[keepaspectratio, scale=0.45]
      {images/direction.png}
 \caption{Target direction}
 \label{Fig:direction}
\end{figure}

学習器には, 上記の3つの目標方向を要素数3, 次元数1のint型の配列で表現した”目標方向指令”を入力する. 目標方向指令のデータ形式を\tabref{table:direction}に示す.

\begin{table}[hbtp]
  \caption{Target direction list}
  \label{table:direction}
  \centering
  \begin{tabular}{|c|c|c|c|}
    \hline
    Target Direction  & Go straight & Turn left & Trun right\\
    \hline
    Data & [100, 0, 0] & [0, 100, 0] & [0, 0, 100]\\
    \hline
  \end{tabular}
\end{table}

% \begin{figure}[hbtp]
%   \centering
%  \includegraphics[keepaspectratio, scale=0.45]
%       {images/direction.png}
%  \caption{Learning phase system of proposed method}
%  \label{Fig:direction}
% \end{figure}

% \subsubsection{etc...}
\newpage

%!TEX root = ../thesis.tex

\section{ネットワーク構造}
提案手法で用いた学習器のネットワークを\figref{Fig:network_structure}に示す. また, ハイパーパラメータについて\tabref{table:param1}に示す. 64x48のRGB画像を入力とする入力層1層, 畳み込み層3層, 全結合層2層を持つ6層のCNNと, このCNNの出力と目標方向指令を入力する入力層1層, 全結合層2層, 出力層1層の全10層から構成されている. 出力はヨー方向の角速度である.

\begin{figure}[hbtp]
  \centering
 \includegraphics[keepaspectratio, scale=0.43]
      {images/network_structure2.png}
 \caption{Structure of network}
 \label{Fig:network_structure}
\end{figure}

\begin{table}[hbtp]
  \caption{Parameters of network}
  \label{table:param1}
  \centering
  \begin{tabular}{|c|c|}
    \hline
    Input data & Image(64x48 pixels, RGB channels), Target direction \\
    \hline
    Optimizer & Adam($alpha = 0.001, beta1 = 0.9, beta2 =  0.999, eps = 1e^{-2}$)\\
    \hline
    Loss function & Softmax-cross-entropy\\
    \hline
    Output data & Angular velocity\\
    \hline
  \end{tabular}
\end{table}

% \begin{figure}[hbtp]
%   \centering
%  \includegraphics[keepaspectratio, scale=0.45]
%       {images/network_structure.png}
%  \caption{Learning phase system of proposed method}
%  \label{Fig:network_structure}
% \end{figure}

% \subsubsection{etc...}
\newpage

%

\chapter{学習時間の短縮化の提案}
\label{chap:experiments}
% この章では, 1節で我々が行ってきた研究\cite{mech}の実験(以下, 「従来の実験」と称する)を実環境に移行する際に, 新たに顕在化した課題点について述べる. また, 課題を解決するための2つのアプローチを提案し, 実験と検証を行う. 2節では, 実験に簡易的なシミュレータを用いる問題点を述べ, 解決策を提示する. 3節では, 実環境で実験を行い, 実環境における提案手法の有効性を検証する.  
この章では, 1節で, 前章で問題となった学習時間の長さについて議論する. 2節では, 前章の実験を基に, 学習ステップ数を減らした実験を行う. 3, 4節では, 問題を解決するための2つのアプローチをそれぞれ提案し, 実験と検証を行う. 5節では, 実験に簡易的なシミュレータを用いる問題点を述べ, 解決策を提示する. 6節では, 実環境で実験を行い, 実環境における提案手法の有効性を検証する.  
%
%\input{experiments/preface}
%
% %!TEX root = ../thesis.tex

\section{実験要件}
実験には下記のコンピュータとソフトウェアを用いた. ロボットモデルは前報\cite{okada1}\cite{okada2}と同様, \figref{Fig:waffle_pi}に示すように, TurtleBot3 Waffle\_piへ3つのカメラを追加したモデルを用いる.

\begin{figure}[hbtp]
  \centering
 \includegraphics[keepaspectratio, scale=0.3]
      {images/Waffle_pi.png}
 \caption{TurtleBot3 waffle\_pi with 3 cameras}
 \label{Fig:waffle_pi}
\end{figure}

\begin{enumerate}
  \item コンピュータ\\
  OS: Ubuntu 20.04 LTS\\
  ROS: Noetic\\
  CPU: intel Core i7-10700F(4.8GHz/8コア/16スレッド)\\
  DRAM: 32GB DDR4(3200/8GB×4)
  \item nav\_cloning(学習器, 統合環境)\\
  \url{https://github.com/open-rdc/nav_cloning}
  \item waypoint\_nav(移動目標地点, 目標方向を出力)\\
  \url{https://github.com/open-rdc/waypoint_nav}
  \item turtlebot3 関連\\
  \url{https://github.com/open-rdc/turtlebot3}
  \item navigation(ナビゲーションパッケージ)\\
  \url{https://github.com/ros-planning/navigation}
\end{enumerate}
% \newpage

%!TEX root = ../thesis.tex

\section{課題点と2つのアプローチによる実験}
我々が行ってきた研究では, 簡易的なシミュレータ上で提案手法が有効だと確認されている. そのため, 次の段階として実環境における提案手法の有効性を検証することを試みた. そこで, 新たに顕在化した課題点は以下の2つの点である.

\begin{itemize}
  \item 実験条件(主に光である)を揃える関係上, 実験を行う時間帯を光の変化が少ない夜間に固定する必要があるため, 1日に実験を行える時間が少なく, 1回の学習に何日も費やす必要がある
  \item 長時間の学習に耐えられるだけのバッテリ容量がロボットにない
\end{itemize}

これらの課題点から, 学習時間の短縮が必要であると判断した. そのため, 2つのアプローチを提案し, 学習量を削減する.
\par
この節では, まず, 従来の実験を簡単に紹介する. 次に, 2つのアプローチについての詳細と行った実験を述べる. 最後に, アプローチを試みる前と各アプローチによる実験結果を比較し, 議論を行う.

\subsection{従来の実験}

\begin{itemize}
  \item 実験環境

  \begin{figure}[hbtp]
  \centering
 \includegraphics[keepaspectratio, scale=0.35]
      {images/tsudanuma2-3_sim.png}
 \caption{Experimental environment from \cite{mech}}
 \label{Fig:tsudanuma2-3_sim}
\end{figure}

\begin{figure}[hbtp]
  \centering
 \includegraphics[keepaspectratio, scale=0.35]
      {images/select_patarn.png}
 \caption{Selecting a path at the T-junction from \cite{mech}}
 \label{Fig:select_patarn}
\end{figure}

  \item 実験方法
  \item 実験結果
\end{itemize}

% \begin{figure}[hbtp]
%   \centering
%  \includegraphics[keepaspectratio, scale=0.8]
%       {images/RaspberryPiMouse.png}
%  \caption{Example}
%  \label{Fig:Example}
% \end{figure}

% \newpage

%!TEX root = ../thesis.tex

\section{実環境に似たシミュレータ環境による実験}
これまでの実験では, 簡易的なシミュレータ環境を用いてきた. しかし, 実験に簡易的なシミュレータ環境を用いるには問題があり, 以下の点である.

\begin{itemize}
  % \item 屋内を模している環境であるため, 影は少ないはずだが, \figref{Fig:sim_shadow}に示すように全体的に影が写り込んでしまっている

  % \begin{figure}[hbtp]
  %   \centering
  %  \includegraphics[keepaspectratio, scale=0.37]
  %       {images/sim_up.png}
  %  \caption{Ratio of data by distance from target path in test phase}
  %  \label{Fig:sim_up}
  % \end{figure} 

  % \begin{figure}[h]
  %   \centering
  %   \begin{minipage}[b]{67mm}
  %     \centering
  %     \includegraphics[width=67mm, height=50.5mm]{images/sim_robot.png}
  %     \caption*{(a)}
  %   \end{minipage} 
  %   % \newpage
  %   % \hspace{0.03\columnwidth}
  %   \begin{minipage}[b]{67mm}
  %     \centering
  %     \includegraphics[width=67mm, height=50.5mm]{images/sim_up.png}
  %     \caption*{(b)}
  %   \end{minipage}
  %   \caption{Number of data per command per 10000steps in conventional experiments}
  %   \label{Fig:sim_shadow}
  % \end{figure}

  \item \figref{Fig:sim_shadow} (a)に示すように, 環境の大半が灰色や白のみで構成されているため, \figref{Fig:sim_shadow} (b)のように視覚による特徴が乏しい. 
  
   \begin{figure}[h]
    \centering
    \begin{minipage}[b]{67mm}
      \centering
      \includegraphics[width=50mm, height=36mm]{images/sim_up.png}
      \caption*{(a) A bird's eye view of the robot}
    \end{minipage} 
    % \newpage
    % \hspace{0.03\columnwidth}
    \begin{minipage}[b]{67mm}
      \centering
      \includegraphics[width=50mm, height=36mm]{images/sim_robot.png}
      \caption*{(b) Robot Perspective}
    \end{minipage}
    \caption{Simple simulator environment}
    \label{Fig:sim_shadow}
  \end{figure}

\end{itemize}

この問題は, 学習フェーズにおいて取得するカメラ画像で学習器を訓練する際に, 大きな影響を及ぼす可能性がある. そのため, ロボットの視覚であるカメラ画像で, より多くの視覚的特徴をとらえるために, \figref{Fig:real_sim}に示すように実環境に似たシミュレータ環境を作成した. 

\begin{figure}[h]
  \centering
  \begin{minipage}[b]{67mm}
    \centering
    \includegraphics[width=50mm, height=36mm]{images/real_sim_up.png}
    \caption*{(a) A bird's eye view of the robot}
  \end{minipage} 
  % \newpage
  % \hspace{0.03\columnwidth}
  \begin{minipage}[b]{67mm}
    \centering
    \includegraphics[width=50mm, height=36mm]{images/real_sim_robot.png}
    \caption*{(b) Robot Perspective}
  \end{minipage}
  \caption{Simulator environment similar to real environment}
  \label{Fig:real_sim}
\end{figure}

% \begin{figure}[hbtp]
%   \centering
%  \includegraphics[keepaspectratio, scale=0.8]
%       {images/RaspberryPiMouse.png}
%  \caption{Example}
%  \label{Fig:Example}
% \end{figure}

\newpage

%!TEX root = ../thesis.tex

\section{実環境の実験}
これまでの節では, いずれもシミュレータ上での実験を行ってきたが, 実験を実環境に移す. それから, 実環境における提案手法の有効性を検証する.

\subsection{実験装置(実環境)}
\begin{itemize}
  \item ロボット
  
  ロボットは前報\cite{okada1}と同様, \figref{Fig:gamma}に示すように, 3つのカメラを搭載したロボットを用いる.

  \vspace{2cm}
  
  \begin{figure}[hbtp]
    \centering
   \includegraphics[keepaspectratio, scale=0.7]
        {images/gamma2.png}
   \caption{Experimental setup from \cite{okada1}}
   \label{Fig:gamma}
  \end{figure}

  \newpage

  \item 環境

  \figref{Fig:real_environment}に示すような千葉工業大学津田沼キャンパス2号館3階で実験を行う.

  \begin{figure}[h]
    \centering
    \begin{minipage}[b]{120mm}
      \centering
      \includegraphics[width=40mm]{images/real.png}
      \caption*{(a) One place in the real environment}
    \end{minipage} 
    % \newpage
    % \hspace{0.03\columnwidth}
    \begin{minipage}[b]{120mm}
      \centering
      \includegraphics[width=95mm]{images/tsudanuma_structure.png}
      \caption*{(b) structure}
    \end{minipage}
    \caption{Real environment}
    \label{Fig:real_environment}
  \end{figure}
\end{itemize}

\subsection{実験方法}
4.2の実験により, 20000stepほど学習させることで, 十分な成功率が得られる可能性が高いことがわかっている. しかし, 実環境で行う実験時間を削減するため, 実環境に似たシミュレータ上で訓練した学習器をファインチューニングする. なお, 本論文ではファインチューニングによる実験結果の影響を議論しない. 実環境における実験の流れを以下に示す.
\begin{enumerate}
  \item 事前に4.3の実験に倣って, シミュレータ上で10000step学習させる. 
  \item 前段階の学習器を用いて初期値を設定し, 実環境で4.1.2 で示した経路を繰り返し走行させる.
  \item 学習を10000step実行後, テストフェーズに移行する. テストフェーズで正しい順序で経路を選択し, 走行を行えるか確認する.
\end{enumerate}
この一連の流れを 10 回繰り返し行う.

\subsection{実験結果}
実験結果を \figref{Fig:real_result} に示す. この図は, それぞれの走行パターンにおいて正しく経路を選し, 走行できた回数を表している. \tabref{table:real} に実験ごとに全パターンを合計した結果を示す. \tabref{table:real} に示すように, 目標方向に従って 78/120 回, 正しい経路を選択する様子が見られた.

\begin{figure}[hbtp]
  \centering
 \includegraphics[keepaspectratio, scale=0.5]
      {images/real_result.png}
 \caption{Experimental setup from \cite{okada1}}
 \label{Fig:real_result}
\end{figure}

\begin{table}[hbtp]
  \caption{Experimental results}
  \label{table:real}
  \centering
  \begin{tabular}{|c|c|c|}
    \hline
    Experiments & Step & Total results\\
    \hline
    Approach1+2 & 20000 & /120(\%)\\
    \hline
    Real & 20000 & 78/120(65\%)\\
    \hline
  \end{tabular}
\end{table}

\newpage

%

\chapter{結論}
\label{chap:end}
本研究では,経路追従行動をカメラ画像を用いた end-to-end 学習で模倣する岡田らの従来手法をベースに,データセットと学習器の入力へ目標方向を加えることで,経路選択をする機能の追加を提案した. また, シミュレータ上での実験を実環境に移す際に, 問題となった学習時間の長さについて, 2つのアプローチを試みることで学習時間を大幅に削減した. 加えて, 実環境での実験を行い, 有効性の検証を行った. 実験結果より, 学習器へ目標方向を与えることで, 指定した経路へ走行する挙動が確認できた.

%\input{experiments/preface}
%
% %!TEX root = ../thesis.tex

\section{}

%

%ここにディレクトリのパスを追加していく
%
%% Back Matter
\backmatter{}
%
\chapter{提案手法}
\label{chap:suggest}
%
本章では, 従来手法をベースとする提案手法についての概要, 提案手法における学習フェーズ, テストフェーズ, 目標方向, ネットワーク構造についての5節に分けて述べる.
%\input{introduction/preface}
%
%!TEX root = ../thesis.tex

\section{提案手法の概要}
従来手法で用いていたデータセットと学習器の入力へ, 「直進」「左折」などの目標方向を追加する. これにより, 学習器の出力による自律移動において, 経路を選択する機能の追加を行った. なお, 追加した要素以外は従来手法と同様である.

% \subsection{RoboCup}

% \begin{figure}[hbtp]
%   \centering
%  \includegraphics[keepaspectratio, scale=0.4]
%       {images/deeplearning_model.png}
%  \caption{Neural network}
%  \label{Fig:Neural network}
% \end{figure}

% \subsubsection{etc...}
\newpage

%!TEX root = ../thesis.tex

\section{学習フェーズ}
提案手法で用いる学習フェーズのシステムを\figref{Fig:suggest_learning_sys}に示す. 自律移動を行う地図を用いたルールベース制御器から目標方向を生成し, データセットに加えている. 
なお, 厳密にはルールベース制御を構成するwaypoint\_navにより目標方向を生成している. 
提案手法では, 
% \figref{Fig:overview}に示すように
LiDARとオドメトリを入力とする地図を用いたルールベース制御器による自律移動を, カメラ画像と目標方向を用いて模倣学習する.

% \vspace{3cm}
\begin{figure}[hbtp]
     \centering
    \includegraphics[keepaspectratio, scale=0.45]
         {images/conventional.png}
    \caption{Learning phase system of conventional method}
    \label{Fig:conventional}
   \end{figure}   

\begin{figure}[hbtp]
  \centering
 \includegraphics[keepaspectratio, scale=0.45]
      {images/suggest_learning_sys2.png}
 \caption{Learning phase system of proposed method}
 \label{Fig:suggest_learning_sys}
\end{figure}

% \begin{figure}[hbtp]
%   \centering
%  \includegraphics[keepaspectratio, scale=0.6]
%       {images/overview.png}
%  \caption{Overview learning phase}
%  \label{Fig:overview}
% \end{figure}

% \subsubsection{etc...}
\newpage

%!TEX root = ../thesis.tex

\section{テストフェーズ}
提案手法におけるテストフェーズでは\figref{Fig:suggest_test_phase}で示すように, 学習器の入力へ目標方向を加えた. なお, テストフェーズも学習フェーズと同様に, 地図を用いたルールベース制御器から目標方向を生成している. 本来ならば, 目的地までカメラ画像のみで自律移動するためには, 目標方向を画像から自動的に作成する仕組みが必要となる. 
% figに動作の様子を示す. 
カメラ画像と目標方向を入力した学習器の出力による自律走行時, 目標方向によって任意の経路を選択する.

\vspace{3cm}

\begin{figure}[hbtp]
  \centering
 \includegraphics[keepaspectratio, scale=0.45]
      {images/suggest_test_phase.png}
 \caption{Learning phase system of proposed method}
 \label{Fig:suggest_test_phase}
\end{figure}

% \subsubsection{etc...}
\newpage

%!TEX root = ../thesis.tex

\section{目標方向}
本研究で用いた目標方向と, そのデータ形式である目標方向指令について述べる. 目標方向を\figref{Fig:direction}に示す. 経路と分岐路において「道なり」に走行(Go straight), 分岐路において「直進(Go straight)」, 「左折(Turn left)」, 「右折(Turn right)」の3つとする.

\begin{figure}[hbtp]
  \centering
 \includegraphics[keepaspectratio, scale=0.45]
      {images/direction.png}
 \caption{Target direction}
 \label{Fig:direction}
\end{figure}

学習器には, 上記の3つの目標方向を要素数3, 次元数1のint型の配列で表現した”目標方向指令”を入力する. 目標方向指令のデータ形式を\tabref{table:direction}に示す.

\begin{table}[hbtp]
  \caption{Target direction list}
  \label{table:direction}
  \centering
  \begin{tabular}{|c|c|c|c|}
    \hline
    Target Direction  & Go straight & Turn left & Trun right\\
    \hline
    Data & [100, 0, 0] & [0, 100, 0] & [0, 0, 100]\\
    \hline
  \end{tabular}
\end{table}

% \begin{figure}[hbtp]
%   \centering
%  \includegraphics[keepaspectratio, scale=0.45]
%       {images/direction.png}
%  \caption{Learning phase system of proposed method}
%  \label{Fig:direction}
% \end{figure}

% \subsubsection{etc...}
\newpage

%!TEX root = ../thesis.tex

\section{ネットワーク構造}
提案手法で用いた学習器のネットワークを\figref{Fig:network_structure}に示す. また, ハイパーパラメータについて\tabref{table:param1}に示す. 64x48のRGB画像を入力とする入力層1層, 畳み込み層3層, 全結合層2層を持つ6層のCNNと, このCNNの出力と目標方向指令を入力する入力層1層, 全結合層2層, 出力層1層の全10層から構成されている. 出力はヨー方向の角速度である.

\begin{figure}[hbtp]
  \centering
 \includegraphics[keepaspectratio, scale=0.43]
      {images/network_structure2.png}
 \caption{Structure of network}
 \label{Fig:network_structure}
\end{figure}

\begin{table}[hbtp]
  \caption{Parameters of network}
  \label{table:param1}
  \centering
  \begin{tabular}{|c|c|}
    \hline
    Input data & Image(64x48 pixels, RGB channels), Target direction \\
    \hline
    Optimizer & Adam($alpha = 0.001, beta1 = 0.9, beta2 =  0.999, eps = 1e^{-2}$)\\
    \hline
    Loss function & Softmax-cross-entropy\\
    \hline
    Output data & Angular velocity\\
    \hline
  \end{tabular}
\end{table}

% \begin{figure}[hbtp]
%   \centering
%  \includegraphics[keepaspectratio, scale=0.45]
%       {images/network_structure.png}
%  \caption{Learning phase system of proposed method}
%  \label{Fig:network_structure}
% \end{figure}

% \subsubsection{etc...}
\newpage

%

%

\end{document}
