\documentclass[uplatex, a4paper, 12pt, openany, oneside]{jsbook}

\usepackage[dvipdfmx]{graphicx}
\usepackage[dvipdfmx]{color}
\usepackage[dvipdfmx, bookmarks=true, setpagesize=false, hidelinks]{hyperref}
\usepackage{pxjahyper}

\usepackage{thesis}
\usepackage{here}
\usepackage{url}
\usepackage{amsmath}
\usepackage{amssymb}
\usepackage{amsfonts}
\usepackage{subcaption}
\usepackage{appendix}
% \usepackage{CJKutf8}

\thesis{修 士 論 文}
\title{
  \centering
    \scalebox{1.0}{移動ロボットのための深層学習を用いた}
    \vspace{-0.3zh}
    \scalebox{1.0}{歩行者の位置予測とナビゲーションへの応用}
    \vspace{1zh}
    \scalebox{0.63}{Pedestrian Position Prediction Using Deep Learning for Mobile Robots}\\
    \vspace{-0.6zh}
    \scalebox{0.63}{and Its Application to Navigation}
    \vspace{-4zh}
}
\setlength{\textwidth}{\fullwidth}
\setlength{\evensidemargin}{\oddsidemargin}

\date{\today}
\vspace{-15.0zh}
\teacher{林原 靖男 教授}
\vspace{-15.0zh}
\organization{千葉工業大学 先進工学研究科 未来ロボティクス専攻}
\author{23S1030 藤原柾}
\vspace{-15.0zh}

\renewcommand{\baselinestretch}{1.2}
\begin{document}

%% Front Matter
\frontmatter{}
%
\chapter{提案手法}
\label{chap:suggest}
%
本章では, 従来手法をベースとする提案手法についての概要, 提案手法における学習フェーズ, テストフェーズ, 目標方向, ネットワーク構造についての5節に分けて述べる.
%\input{introduction/preface}
%
%!TEX root = ../thesis.tex

\section{提案手法の概要}
従来手法で用いていたデータセットと学習器の入力へ, 「直進」「左折」などの目標方向を追加する. これにより, 学習器の出力による自律移動において, 経路を選択する機能の追加を行った. なお, 追加した要素以外は従来手法と同様である.

% \subsection{RoboCup}

% \begin{figure}[hbtp]
%   \centering
%  \includegraphics[keepaspectratio, scale=0.4]
%       {images/deeplearning_model.png}
%  \caption{Neural network}
%  \label{Fig:Neural network}
% \end{figure}

% \subsubsection{etc...}
\newpage

%!TEX root = ../thesis.tex

\section{学習フェーズ}
提案手法で用いる学習フェーズのシステムを\figref{Fig:suggest_learning_sys}に示す. 自律移動を行う地図を用いたルールベース制御器から目標方向を生成し, データセットに加えている. 
なお, 厳密にはルールベース制御を構成するwaypoint\_navにより目標方向を生成している. 
提案手法では, 
% \figref{Fig:overview}に示すように
LiDARとオドメトリを入力とする地図を用いたルールベース制御器による自律移動を, カメラ画像と目標方向を用いて模倣学習する.

% \vspace{3cm}
\begin{figure}[hbtp]
     \centering
    \includegraphics[keepaspectratio, scale=0.45]
         {images/conventional.png}
    \caption{Learning phase system of conventional method}
    \label{Fig:conventional}
   \end{figure}   

\begin{figure}[hbtp]
  \centering
 \includegraphics[keepaspectratio, scale=0.45]
      {images/suggest_learning_sys2.png}
 \caption{Learning phase system of proposed method}
 \label{Fig:suggest_learning_sys}
\end{figure}

% \begin{figure}[hbtp]
%   \centering
%  \includegraphics[keepaspectratio, scale=0.6]
%       {images/overview.png}
%  \caption{Overview learning phase}
%  \label{Fig:overview}
% \end{figure}

% \subsubsection{etc...}
\newpage

%!TEX root = ../thesis.tex

\section{テストフェーズ}
提案手法におけるテストフェーズでは\figref{Fig:suggest_test_phase}で示すように, 学習器の入力へ目標方向を加えた. なお, テストフェーズも学習フェーズと同様に, 地図を用いたルールベース制御器から目標方向を生成している. 本来ならば, 目的地までカメラ画像のみで自律移動するためには, 目標方向を画像から自動的に作成する仕組みが必要となる. 
% figに動作の様子を示す. 
カメラ画像と目標方向を入力した学習器の出力による自律走行時, 目標方向によって任意の経路を選択する.

\vspace{3cm}

\begin{figure}[hbtp]
  \centering
 \includegraphics[keepaspectratio, scale=0.45]
      {images/suggest_test_phase.png}
 \caption{Learning phase system of proposed method}
 \label{Fig:suggest_test_phase}
\end{figure}

% \subsubsection{etc...}
\newpage

%!TEX root = ../thesis.tex

\section{目標方向}
本研究で用いた目標方向と, そのデータ形式である目標方向指令について述べる. 目標方向を\figref{Fig:direction}に示す. 経路と分岐路において「道なり」に走行(Go straight), 分岐路において「直進(Go straight)」, 「左折(Turn left)」, 「右折(Turn right)」の3つとする.

\begin{figure}[hbtp]
  \centering
 \includegraphics[keepaspectratio, scale=0.45]
      {images/direction.png}
 \caption{Target direction}
 \label{Fig:direction}
\end{figure}

学習器には, 上記の3つの目標方向を要素数3, 次元数1のint型の配列で表現した”目標方向指令”を入力する. 目標方向指令のデータ形式を\tabref{table:direction}に示す.

\begin{table}[hbtp]
  \caption{Target direction list}
  \label{table:direction}
  \centering
  \begin{tabular}{|c|c|c|c|}
    \hline
    Target Direction  & Go straight & Turn left & Trun right\\
    \hline
    Data & [100, 0, 0] & [0, 100, 0] & [0, 0, 100]\\
    \hline
  \end{tabular}
\end{table}

% \begin{figure}[hbtp]
%   \centering
%  \includegraphics[keepaspectratio, scale=0.45]
%       {images/direction.png}
%  \caption{Learning phase system of proposed method}
%  \label{Fig:direction}
% \end{figure}

% \subsubsection{etc...}
\newpage

%!TEX root = ../thesis.tex

\section{ネットワーク構造}
提案手法で用いた学習器のネットワークを\figref{Fig:network_structure}に示す. また, ハイパーパラメータについて\tabref{table:param1}に示す. 64x48のRGB画像を入力とする入力層1層, 畳み込み層3層, 全結合層2層を持つ6層のCNNと, このCNNの出力と目標方向指令を入力する入力層1層, 全結合層2層, 出力層1層の全10層から構成されている. 出力はヨー方向の角速度である.

\begin{figure}[hbtp]
  \centering
 \includegraphics[keepaspectratio, scale=0.43]
      {images/network_structure2.png}
 \caption{Structure of network}
 \label{Fig:network_structure}
\end{figure}

\begin{table}[hbtp]
  \caption{Parameters of network}
  \label{table:param1}
  \centering
  \begin{tabular}{|c|c|}
    \hline
    Input data & Image(64x48 pixels, RGB channels), Target direction \\
    \hline
    Optimizer & Adam($alpha = 0.001, beta1 = 0.9, beta2 =  0.999, eps = 1e^{-2}$)\\
    \hline
    Loss function & Softmax-cross-entropy\\
    \hline
    Output data & Angular velocity\\
    \hline
  \end{tabular}
\end{table}

% \begin{figure}[hbtp]
%   \centering
%  \includegraphics[keepaspectratio, scale=0.45]
%       {images/network_structure.png}
%  \caption{Learning phase system of proposed method}
%  \label{Fig:network_structure}
% \end{figure}

% \subsubsection{etc...}
\newpage

%

%
%% Main Matter
\mainmatter{}
%
\chapter{序論}
\label{chap:introduction}
%
%\input{introduction/preface}
%
%!TEX root = ../thesis.tex

\section{背景}
近年,ロボット技術の進歩に伴い,製造業,物流,サービス業など幅広い分野で自律移動ロボットの活用が進んでいる.人間とロボットが共存する環境では,ロボットが安全かつ効率的に動作するために,周囲の環境,特に人間の行動を正確に理解し,適切に対応することが求められる.なかでも歩行者は動的な障害物であり,その予測困難な動きはロボットのナビゲーションにとって大きな課題となっている.

この課題に対して,Social Force Model\cite{s-lstm}などの物理法則に基づくルールベース手法が提案されてきた.これらの手法では,歩行者の動きを物理的な力の相互作用によってモデル化し,歩行者同士や障害物との関係を考慮することで,より現実的な動きを再現できるように試みている.しかし,複雑な環境や多様な人間の行動を完全に表現することは依然として困難である.

一方,近年では深層学習による歩行者の動きを予測する手法が注目されている.特に,リカレントニューラルネットワーク(RNN)\cite{rumelhart1986learning1,rumelhart1986learning2}やその発展形である長短期記憶(LSTM)\cite{hochreiter1997long}ネットワークを用いたアプローチが有望視されている.これらの手法は,歩行者の過去の軌跡を考慮して未来の動きを予測できる.また,グラフ畳み込みニューラルネットワーク(GCN)\cite{kipf2016semi-gcn}やグラフアテンションネットワーク(GAT)\cite{velickovic2017graph-gat}を用いた手法も注目を集めている.これらの手法は,個々の歩行者をノード,相対距離などの関係性をエッジとみなしてグラフ化することで,歩行者間の相互作用を効率的に学習できる.

しかし,これらの深層学習ベースの手法を移動ロボットの文脈に直接適用した研究はまだ少ない.また,移動ロボットのナビゲーションに予測結果を応用した例の報告が少ない.さらに,移動ロボットを対象とする既存の研究でも,歩行者へのセンサ取り付けなど環境への介入が必要だったり,ロボットのナビゲーション性能を評価するための実験環境が十分に整備されていない場合が多い.

また,移動ロボットのナビゲーションに予測結果を応用する場合,歩行者同士の相互作用の他にロボットと人間との間にも相互作用が存在する.そのため,歩行者同士の相互作用だけでなく,ロボットと歩行者の相互作用も考慮する必要がある点も課題として挙げられる.つまり,ロボットの行動が歩行者の行動に影響を与えるため,歩行者のみならずロボットの行動も考慮した軌道予測が求められる.

そこで本研究では,上記2点の課題を克服するため,ロボットの行動を考慮した歩行者の軌道予測を行い,その予測結果を用いたナビゲーションシステムを提案する.そして,本システムによるナビゲーション性能を実験を通じて検証し,その有効性を示すことを目指す.

\newpage

%!TEX root = ../thesis.tex

\section{関連研究}
歩行者の複雑な動きを予測するために,深層学習を応用しようという研究が,近年注目を集めている.

\subsection{LSTMによる歩行者の軌道予測の応用事例}
Alexandreらは,人間の動きを学習し,未来の軌跡を予測できるLSTMモデルを提案している.この研究では,

\subsection{GNNによる歩行者の軌道予測の応用事例}
hoge

\subsection{ロボットの将来の行動を軌道予測に用いた事例}
丹野らは,歩行者の軌道予測に過去の動きだけでなく,ロボットが選択する将来の動きを考慮して予測をしている.実験では,ロボットの将来の行動を考慮しない場合と比較して,ロボットの動きに合わせて変化する歩行者の動きを予測できている.

\subsubsection{etc...}
\newpage

%!TEX root = ../thesis.tex

\section{目的}
% \ref{sec:relate-research}節で紹介した深層学習によるアプローチは,歩行者の複雑な動きを予測する上で有用である.しかし,環境への介入なしに予測結果を移動ロボットのナビゲーションへ応用している研究事例は少ない.また,歩行者の軌道予測を移動ロボットで活用するにあたっては,ロボットと歩行者間の相互作用も考慮すべきである.
% しかし,ロボットと人間を区別せずにエンコードして,より簡素なモデル構造
% そこで本論文では,環境への介入を要しないかたちでロボットの行動を考慮した歩行者の軌道予測手法を提案する.さらに,本手法を移動ロボットのナビゲーションに応用し,ロボットが安全かつ効率的に移動できるか検討する.

% ロボットの行動を考慮した歩行者の軌道予測を行い,その予測結果を用いたナビゲーションシステムを提案する.そして,本システムによるナビゲーション性能を実験を通じて検証し,その有効性を示すことを目指す.

本論文では,ロボットの行動を考慮した歩行者の軌道予測を可能にする深層学習モデルを提案する.さらに,提案モデルを移動ロボットのナビゲーションに応用し,実験によってナビゲーション性能を検証することで,その有効性を示すことを目指す.


%!TEX root = ../thesis.tex

\section{本論文の構成}
本論文は,全7章で構成されている.第1章では,研究の背景や関連研究,目的を述べた.第2章では,本研究に用いた要素技術について述べる.第3章では,本研究で提案する手法について具体的に説明し,設計したネットワークの基本性能を確認した.第4章では,提案手法の有効性を検証する実験について述べる.第5章では,予測結果を用いたナビゲーションシステムの構築について述べる.第6章では,実験で予測結果を利用したナビゲーションの有効性を検証する.第7章では,本研究の成果をまとめ,今後の展望について述べる.

\newpage

%

\chapter{要素技術}
\label{chap:element_technology}
%
%\input{introduction/preface}
%
%!TEX root = ../thesis.tex

\section{ナビゲーション}\label{sec:navigation-stack}
LiDARやオドメトリなどのセンサ情報を利用して自律走行する移動ロボットの多くは,ROS Navigation stack\cite{nav1,nav2}を採用している.例えば,つくば市内の遊歩道を移動ロボットが自律走行する技術チャレンジ「つくばチャレンジ」において,原らの技術調査\cite{robomech2024-hara}によると,少なくとも参加76チーム中22チームが自律走行でROS Navigation stackを使用していたと報告されている.これは,オープンソースのソフトウェアとしては最多の利用数であった.ROS Navigation stackには,以下の主要な機能が含まれている.

\begin{itemize}
     \item \textbf{自己位置推定(Localization)}\\
     AMCL(Adaptive Monte Carlo Localization)アルゴリズムを用いて,ロボットの現在位置を推定する.
     \item \textbf{地図生成(Mapping)}\\
     SLAM(Simultaneous Localization and Mapping)技術を活用し,環境のマッピングを行う.
     \item \textbf{経路計画(Path Planning)}\\
     Dijkstra法\cite{dijkstra2022note}やA*アルゴリズム\cite{hart1968formal-astar}を用いて,最適経路を計画する.
     \item \textbf{障害物回避(Obstacle Avoidance)}\\
     センサデータに基づいて障害物を検知し,動的に回避しながら移動する.
\end{itemize}

\newpage

%

\chapter{提案手法}
\label{chap:proposed_method}
%
%\input{introduction/preface}
%
%!TEX root = ../thesis.tex

\section{本章の概要}

本章では,2章で述べたようにロボットの行動を考慮した歩行者の軌道予測を行う手法を提案する.学習に用いたアルゴリズムに関する内容を3.2章,

% 以下では,予測を行うネットワーク構造とその学習方法に関して説明する.

% 〜について目指す.〜について提案する.

\section{問題設定}
本論文における歩行者のグラフ表現を説明する.時刻$t$におけるシーン内の歩行者の位置を表す空間グラフを$G_t = (V_t, E_t)$と定義する.ここで,$V_t = \{ v^i_t \mid i = 1, \dots N \}$はグラフ$G_t$のノードの集合であり,各ノード$v^i_t$は$i$番目の歩行者の位置$(x^i_t, y^i_t)$を属性として持つ.$E_t = \{ e^{ij}_t \mid i, j = 1, \dots N \}$はグラフ$G_t$のエッジの集合であり,$e^{ij}_t$はノード$v^i_t$と$v^j_t$が接続されている場合1,そうでない場合0の値をとる.

人間の軌道予測は,歩行者の将来の2次元空間の$x,y$座標を,事前観測の情報を与えて予測ステップ分出力することである.
ここで,歩行者$i$の各時刻の位置を
\begin{equation}
  V_t = \{ v^i_t = (x^i_t, y^i_t) \in \mathbb{R}^2 \mid i = 1, \dots N \}
\end{equation}
と表す.また,$t_{obs}$を観測時間としたとき,観測された全歩行者の位置データを
\begin{equation}
  V_{obs} = \{ V_t \mid t = 1, \dots t_{obs} \}  
\end{equation}
と表す.ここで,時刻$t$の2次元空間における歩行者$i$の位置の確率分布を
\begin{equation}
  \hat{V}_t = \{ \hat{p}^i_t = (\hat{x}^i_t, \hat{y}^i_t) \mid i = 1, \dots N \} \label{hat-pos}
\end{equation}
と表す.$t_{pred}$を予測時間としたとき,予測する全歩行者の位置データを
\begin{equation}
  V_{pred} = \{ \hat{V}_t \mid t = t_{obs + 1}, \dots t_{pred} \}
\end{equation}
と表すことができる.$p^i_t$が$\mathcal{N}(\mu^i_t, \sigma^i_t, \rho^i_t)$となる2変量ガウス分布に従うと仮定する.すなわち,$\hat{p}^i_t$は平均$\hat{\mu}^i_t$,分散$\hat{\sigma}^i_t$,相関係数$\hat{\rho}^i_t$を持つガウス分布
\begin{equation}
  \hat{p}^i_t \sim \mathcal{N}(\hat{\mu}^i_t, \hat{\sigma}^i_t, \hat{\rho}^i_t)
\end{equation}
に従う.ここで,平均を$\hat{\mu}^i_t = (\hat{\mu}^i_{x, t}, \hat{\mu}^i_{y, t})$,分散を$\hat{\sigma}^i_t = (\hat{\sigma}^i_{x, t}, \hat{\sigma}^i_{y, t})$とする.

\section{ネットワーク構造}
本手法で用いられるネットワークは,エンコーダ・デコーダ構造で構成されている.ネットワークは,主にエンコーダモジュール,グラフアテンションネットワークモジュール,デコーダモジュールから構成される.エンコーダモジュールでは,入力データから特徴量を抽出し,グラフアテンションネットワークを用いて,ノード間の関係性を学習する.これらの潜在表現を基に,デコーダモジュールで目的のシーケンス長の予測を行う.
\figref{Fig:network}は,このネットワークの概要を表している.

\begin{figure}[hbtp]
  \centering
 \includegraphics[keepaspectratio, scale=0.36]
      {images/network-comp.pdf}
 \caption{Network Structure}
 \label{Fig:network}
\end{figure}   

\subsection{エンコーダモジュール}
エンコーダでは,まず,入力の各時刻における歩行者の過去の位置は,線形変換層$\phi_{emb}$を用いて高次元空間に埋め込まれる.この埋め込み表現は,LSTM層に入力され,歩行者の過去の位置情報を時間的に処理し,隠れ状態を出力する.隠れ状態は,歩行者の過去の動きを要約したものであり,次のグラフアテンションネットワークに渡される.エンコーダでの処理を式\eqref{emb},\eqref{lstm-en}に示す.なお,$v^i_t \text{と} v^t_i$は等価である.
\begin{align}
  e^t_i &= \phi_{emb}(v^t_i ; W_{emb}) \label{emb} \\
  h_i &= \text{LSTM}_{en}(e^t_i, h^{t-1}_i ; W_{en}) \label{lstm-en}
\end{align}

\subsection{グラフアテンションネットワーク}
ネットワークの中核を担うグラフアテンションネットワーク(GAT)\cite{velickovic2017graph-gat}は,複数層で構成される.各GAT層は,マルチヘッドアテンション機構を用いて,各ノードの特徴をその近傍ノードの特徴と集約する.アテンション機構は,ノード間の関係の強さに基づいて重み付けを行うため,より関連性の高いノードからの情報が強調される.本ネットワークでは,複数のGAT層を積み重ねることで,グラフ構造における高次の依存関係を捉えることができる.各タイムステップにおいて,対応する隣接行列を用いてグラフ畳み込みが実行され,動的なグラフ構造の変化にも対応できる.

歩行者$i$の隠れ状態$h_i$が与えられたとき,全ての歩行者に対して,複数のGAT層を適用する.各層は以下のように適用される.式\eqref{gat-eij}は,ノード$i$に対するノード$j$の特徴の重要度を表している.$W$は重み行列であり,$a$は共有アテンション機構である.また,$\sigma$は活性化関数であり,本手法ではネットワーク全体でPReLU\cite{he2015delving-prelu}を用いた.
\begin{align}
  e_{ij} = a(Wh_i, Wh_j) \label{gat-eij} \\
  \alpha_{ij} = \text{softmax}_{j}(e_{ij}) \\
  h'_i = \sigma\Bigg(\sum_{j \in \mathcal{N}_i} \alpha_{ij}Wh_j \Bigg)
\end{align}

\subsection{デコーダモジュール}
GAT層からの出力は,デコーダモジュールによって将来の歩行者の予測位置に変換される.まず,2つ目のLSTM層がGATによって生成された時空間特徴表現をさらに時間的に処理する.次に,1次元畳み込み層を用いて,入力シーケンス長から予測シーケンス長への変換が行われる.この層は,予測時間に合わせて特徴表現を調整する役割を果たす.最後に多層パーセプトロン(MLP)が予測値を生成する.式\eqref{output}のように,ネットワークの最終的な出力は5次元である.なお,$\hat{P}^t_i\text{と}\hat{P}^i_t$は等価である.
\begin{align}
  h''_i = \text{LSTM}_{dec}(h'_i, h_{deci}; W_{dec})\\
  c^t_i = \text{CNN}(h''_i; W_{cnn}) \\
  \hat{P}^t_i = \text{MLP}_{d}(c^t_i; W_{d}) \\
  \hat{P}^i_t = [\hat{\mu}^i_{x,t}, \hat{\mu}^i_{y, t}, \hat{\sigma}^i_{x, t}, \hat{\sigma}^i_{y, t}, \hat{\rho}^i_t] \label{output}
\end{align}

さらに,本ネットワークでは\figref{Fig:network}のように,GAT層とLSTM層の出力にスキップ接続\cite{he2016deep-resnet}を導入することで,勾配消失問題\cite{hochreiter2001gradient-grad,weinleindiplomarbeit-grad, schmidhuber2015deep-grad}を軽減し,学習を安定化させる.このアーキテクチャにより,時系列データの動的な変化とグラフ構造におけるノード間の複雑な関係を効果的に捉え,高精度な予測を実現する.

\section{ロボットの行動を考慮した軌道予測}
% 先行研究\cite{si2023-tanno}と同様に,予測時間においてロボットが詮索する予定の経路を予測に用いることが本研究のテーマだが,

歩行者が頻繁に行き交うような環境において,低速域で走行する移動ロボットの行動が歩行者の行動と近似できるという仮定を前提とする.つまり,ロボットと歩行者を区別せずに予測を行う.
ロボットの行動を考慮した軌道予測を行うために,ロボットの経路情報を追加の入力としてネットワークに与える.具体的には,ロボットの位置情報を歩行者の位置情報と同様にグラフのノードとして扱い,エンコーダモジュールで処理する.これにより,ロボットと歩行者の相互作用を考慮した予測が可能となる.

ロボットの位置を表すノード$r_t$を追加し,グラフ$G_t$を拡張する.拡張されたグラフ$\tilde{G}_t = (\tilde{V}_t, \tilde{E}_t)$は,$\tilde{V}_t = V_t \cup \{ r_t \}$と定義される.エッジ集合$\tilde{E}_t$も同様に拡張され,ロボットと歩行者間のエッジが追加される.なお,拡張されたグラフ$\tilde{G}_t$は,ネットワーク構造を変更せずに入力することができる.

エンコーダモジュールでは,ロボットの位置$r_t$も他のノードと同様に埋め込み表現$e^t_r$に変換され,LSTM層に入力される.グラフアテンションネットワークでは,ロボットと歩行者間の相互作用を考慮したアテンション重みが計算される.デコーダモジュールでは,ロボットの位置情報を含む潜在表現を基に,将来の歩行者の位置を予測する.
このようにして,ロボットの行動を考慮した軌道予測を実現する.4章では,実験で提案手法の有効性を評価する.

\section{学習方法}
提案するネットワークの学習方法の詳細について述べる.
ネットワークは,負の対数尤度を最小化するように学習される.損失関数を式に示す.
\begin{equation}
  L^i = -\sum_{t=t_{obs+1}}^{t_{pred}} \log \left( P(\hat{p}^i_t \mid \hat{\mu}^i_t, \hat{\sigma}^i_t, \hat{\rho}^i_t) \right)
\end{equation}

このネットワークは,2つの歩行者の軌跡データを含むデータセットで学習される.ETHデータセット\cite{pellegrini2009you-eth}とUCYデータセット\cite{lerner2007crowds-ucy}である.この2つのデータセットには,図のような5つのシーンがあり,計1536人の歩行者のデータが含まれている.5つのシーンは,Zara1,Zara2,Univ,Eth,Hotelから構成されている.データセットの軌跡は0.4秒ごとにサンプリングされたものである.

\begin{figure}[hbtp]
  \centering
 \includegraphics[keepaspectratio, scale=0.5]
      {images/RaspberryPiMouse.png}
 \caption{Neural Network}
 \label{Fig:hoge4}
\end{figure}

\newpage

\section{学習環境}
本研究で行う学習は全て以下の\tabref{tab:environment}に示す環境で実施する.

\begin{table}[hbtp]
  \centering
  \caption{Experimental Setup}
  \label{tab:environment}
  \begin{tabular}{ll}
    \hline
    OS & Ubuntu 20.04.6 LTS \\
    CPU & Intel Core i7-10700F \\
    GPU & NVIDIA GeForce RTX 3060 \\
    Memory & 32GB \\
    Language & Python 3.8.10 \\
    Flamework & PyTorch 2.4.1 \\
    \hline
  \end{tabular}
\end{table}

\section{ネットワークの予備実験}
ロボットの行動を考慮した軌道予測が行えるか実験で確認する前に,提案したネットワークの性能を確認するため,予備実験を行った.

\subsection{実験概要}
この実験では,訓練済みモデルを後述する2種類の指標により評価する.その結果を複数のベースラインモデルと比較する.

\subsection{学習条件}
ネットワークの学習は,先行研究\cite{s-lstm,s-stgcnn}と同様の戦略に従い,リーブワンアウト(Leave One Out)方式を採用する.これにより,データセットを最大限に活用することができる.
具体的には,5つのシーンの内,1つのシーンをテストデータとして取り除き,残りのデータを用いてモデルを訓練・検証する.テスト時には,8ステップにあたる3.2秒間観測し,次の12ステップにあたる4.8秒間を予測する.学習パラメータは,バッチサイズを128,Adam\cite{kingma2014adam}を用いて250エポック分の学習を行った.学習率は0.001である.

\subsection{評価指標}
モデルの評価には,以下の2つの指標を用いた.


\subsection{結果}
\subsection{考察}

\newpage

%

\chapter{ロボットの行動を考慮した歩行者の軌道予測の検証実験}
\label{chap:experiments_oculus}
%
%\input{introduction/preface}
%
%!TEX root = ../thesis.tex

% \vspace{-10pt}

\section{本章の概要}
本章では,\ref{sec:oculus-exp-overview}節で実験の概要を示す.また,\ref{sec:oculus-exp-method}節で実験方法,\ref{sec:oculus-exp-result}節で結果と考察について述べる.

\section{実験概要}\label{sec:oculus-exp-overview}
提案手法の有効性を評価するために,現実の環境においてロボットと歩行者がどのように相互作用するかを観察し,提案手法を用いてその行動を予測する実験を行う.実験には20代の男性4名が参加し,その中からランダムに選ばれた2名のデータを予測用に利用した.

\section{実験方法}\label{sec:oculus-exp-method}
本実験は先行研究\cite{si2023-tanno}を参考にして実施したものである.実験は\figref{Fig:oculus-exp-overview}に示す室内環境で行い,\figref{fig:tracking-robot}に示す移動ロボット(ORNE-box2\cite{井口颯人2023屋外自律移動ロボットプラットフォーム-orne})を使用した.実験参加者は\figref{Fig:oculus-exp-overview}に示したように,移動ロボットの正面から歩行を開始し,事前に指定された目的地(A/B/C)のいずれかまでロボットを避けながら歩行する.各参加者には,各目的地への歩行を3回ずつ繰り返してもらった.ロボットの行動が歩行者の将来の軌道にどのように影響するかを観察するため,ロボットは一定の速度0.3m/sで直進した後,次の3つの挙動のいずれかを事前に知らせることなく実行する.

\begin{itemize}
  \item 速度を変えず直進
  \item 速度を変えず右に避ける
  \item 速度を変えず左に避ける
\end{itemize}

\figref{fig:tracking}に示すように,参加者とロボットの位置はOculus Riftの専用コントローラを用いて計測する.また,正確にコントローラをトラッキングするため,2基の赤外線センサを\figref{Fig:oculus-exp-overview}の斜線部に\figref{Fig:oculus-sensor}のように配置する.データは,ETH\cite{pellegrini2009you-eth},UCY\cite{lerner2007crowds-ucy}データセットのフォーマットで取得する.

予測の評価は,\ref{chap:proposed_method}章で行った予備実験と同じ手順で進めた.ロボットの行動を考慮した軌道予測手法の有効性を評価するため,ロボットの行動を考慮する場合としない場合の予測を行い,それぞれの結果を比較する.

\begin{figure}[H]
  \centering
 \includegraphics[keepaspectratio, scale=0.27]
      {images/oculus_experiments.pdf}
\caption{Experimental environment}
 \label{Fig:oculus-exp-overview}
\end{figure} 

\begin{figure}[H]
  \centering
  \begin{minipage}{0.42\textwidth}
    \centering
    \includegraphics[width=\textwidth]{images/tracking-robot.pdf}
    \subcaption{Robot}
    \label{fig:tracking-robot}
  \end{minipage}
  % \hfill
  \begin{minipage}{0.42\textwidth}
    \centering
    \includegraphics[width=\textwidth]{images/tracking-ped.pdf}
    \subcaption{Pedestrian}
    \label{fig:tracking-ped}
  \end{minipage}
  \caption{Tracking sensor setup}
  \label{fig:tracking}
\end{figure}

\begin{figure}[H]
  \centering
 \includegraphics[keepaspectratio, scale=0.32]
      {images/tracking-sensor.pdf}
\caption{Infrared sensor setup}
 \label{Fig:oculus-sensor}
\end{figure}

\section{結果と考察}\label{sec:oculus-exp-result}
\tabref{tab:robot-behavior}に,ロボットの行動を考慮する場合としない場合における各評価指標の値を示す.考慮しない場合に比べて,ADEは7.5%,FDEは15.9%誤差を改善した.
この結果から,ロボットの行動を考慮することで,歩行者の軌道予測の精度が向上することがわかる.特に,FDEの改善が顕著であり,これはロボットの動きが歩行者の最終的な位置に大きな影響を与えることを示唆している.一方で,ADEの改善は比較的小さい.これは,\figref{Fig:oculus-exp-overview}に示したように,3箇所の目的地の内,2箇所で角を曲がる必要があるため,途中の軌道予測が困難であり,評価指標の値の改善が抑えられたと考えられる.

\begin{table}[H]
  \begin{center}
  \caption{Comparison with and without taking into account the robot's behavior}
  \label{tab:robot-behavior}
  % \footnotesize
  \begin{tabular}{c||c|c}
   & ADE & FDE \\ 
  \hline \hline
  Normal      & 0.40       & 0.69                      \\
  \hline
  Considering robot behavior    & 0.37       & 0.58                      \\
  \hline
  \end{tabular}
  \end{center}
\end{table}

\figref{Fig:pred-straight},\figref{Fig:pred-right},\figref{Fig:pred-left}に実験で取得した目的地Aへの歩行データに対する予測例を示す.ロボットの行動を考慮せず予測を行うと,いずれのロボットの行動に対しても避けるような予測は見られなかった.一方,ロボットの行動を考慮した予測では,より真値に近い予測を行う様子が確認できた.また,\figref{Fig:pred-straight},\figref{Fig:pred-right}では,ロボットを避けるような予測をしている.このことから,ロボットの行動を考慮した軌道予測は,人間とロボット間の相互作用を捉えていることが分かる.

\vspace{-10pt}

\begin{figure}[H]
  \centering
 \includegraphics[keepaspectratio, scale=0.58]
      {images/pred_straight.pdf}
\caption{Predicted trajectory when the robot moves straight}
 \label{Fig:pred-straight}
\end{figure}

\begin{figure}[H]
  \centering
 \includegraphics[keepaspectratio, scale=0.58]
      {images/pred_right.pdf}
\caption{Predicted trajectory when the robot moves right}
 \label{Fig:pred-right}
\end{figure}

\vspace{-30pt}

\begin{figure}[H]
  \centering
 \includegraphics[keepaspectratio, scale=0.58]
      {images/pred_left.pdf}
\caption{Predicted trajectory when the robot moves left}
 \label{Fig:pred-left}
\end{figure}

\newpage

%

\chapter{予測結果のナビゲーションへの応用}
\label{chap:application}
%
%\input{introduction/preface}
%
%!TEX root = ../thesis.tex

\section{本章の概要}
本章では,\ref{sec:nav-sys}節でシステムの概要を示す.

% \newpage

\section{システム概要}\label{sec:nav-sys}
\figref{Fig:nav-system}に,ナビゲーションシステムの概要図を示す.主に構成しているモジュールを以下に示す.

\begin{itemize}
  \item \textbf{制御モジュール} \\
  制御モジュールは,Navigation Stackの主要コンポーネントであるmove\_baseで構成されている.
  センサデータと目標位置を受け取り,適切なロボットの制御指令を出力する.
  \item \textbf{認識モジュール} \\
  認識モジュールは,YOLOを用いて歩行者の検出及び,追跡を行う.そして,観測時間分のデータをまとめて時系列データとして,独自メッセージで出力する.
  \item \textbf{予測モジュール} \\
  予測モジュールは,認識モジュールからメッセージを受け取り,そのデータを基に\chapref{chap:proposed_method}で述べたネットワークを用いて歩行者の位置予測を行う.その後,制御モジュールのglobal\_costmapの独自レイヤで予測結果をコストマップに反映する.
\end{itemize}

\begin{figure}[hbtp]
  \centering
 \includegraphics[keepaspectratio, scale=0.77]
      {images/application_system.pdf}
 \caption{Navigation System Overview}
 \label{Fig:nav-system}
\end{figure}  

\section{歩行者の位置推定}

\section{ナビゲーションで予測結果を扱う方法}

\newpage

%

\chapter{予測結果を利用したナビゲーションの実験}
\label{chap:experiments_pred_sim}
%
%\input{introduction/preface}
%
%!TEX root = ../thesis.tex

% \vspace{-1pt}

\section{本章の概要}
本章では,移動ロボットのナビゲーションで予測結果を利用する実験を行い,その結果を評価する.まず,実験の概要と方法について説明し,次に実験シナリオと使用したシミュレータ環境について述べる.最後に,実験結果を踏まえて考察を行う.

\section{実験概要}
予測結果を応用したナビゲーションシステムによって,移動ロボットの挙動がどのように変化するかを観察・評価するための実験を行う.実験では,2つのナビゲーションシナリオで移動ロボットを自律走行させ,各シナリオにつき10回ずつ走行を実施してデータを取得した.

\section{実験方法}
本実験の予測には,\ref{chap:proposed_method}章で提案したネットワークを用いた.ただし,使用したモデルはホールドアウト(Hold Out)方式で学習を行った.それ以外は同様の学習条件である.
また,\ref{chap:application}章で述べたシステムを用いて,ナビゲーションに予測結果を適用した.

実験は\figref{Fig:sim-env}に示すように,Gazebo\cite{Gazebo62:online}のシミュレータ環境上で行う.また,\figref{Fig:sim-robot}に示すように,シミュレータで再現した移動ロボット(ORNE-box2\cite{井口颯人2023屋外自律移動ロボットプラットフォーム-orne})を用い,歩行者は\figref{Fig:sim-actor}に示すGazeboのプラグイン\cite{Actors-G87:online}を使用してシミュレータ内に配置した.

\begin{figure}[H]
  \centering
 \includegraphics[keepaspectratio, scale=0.15]
      {images/sim-env.png}
\caption{Simulator environment}
 \label{Fig:sim-env}
\end{figure} 

\vspace{-10pt}

\begin{figure}[H]
  \centering
  \begin{subfigure}{0.40\textwidth}
    \centering
    \includegraphics[keepaspectratio, scale=0.15]{images/sim-robot.png}
    \caption{Simulated robot}
    \label{Fig:sim-robot}
  \end{subfigure}
  % \hfill
  \begin{subfigure}{0.40\textwidth}
    \centering
    \includegraphics[keepaspectratio, scale=0.15]{images/sim-actor.png}
    \caption{Simulated actor}
    \label{Fig:sim-actor}
  \end{subfigure}
  \caption{Simulated robot and actor}
  \label{Fig:sim-robot-actor}
\end{figure}

% 移動ロボットの自律走行は,つくばチャレンジ2024EX@イーアスつくば\cite{つくばチャレンジ36:online}で千葉工業大学 未来ロボティクス学科 box2チームが完走した際のソフトウェア構成,パラメータを参考にした.これは,Githubで公開されている\cite{openrdco85:online}.なお,後述するシナリオには狭い通路の走行が含まれるため,yaw方向の角速度を小さくなるように変更している.

本研究で用いた移動ロボットの自律走行ソフトウェアおよびパラメータ設定は,つくばチャレンジ2024EX@イーアスつくば\cite{つくばチャレンジ36:online}において,千葉工業大学 未来ロボティクス学科 box2チームが完走した際の構成を参考にしている(Github\cite{openrdco85:online}で公開).なお,後述のシナリオには狭い通路を走行する箇所が含まれているため,yaw方向の角速度を小さくなるように調整している.

\newpage

実験は\figref{Fig:experiment-scenarios}に示す2種類のシナリオで行った.
\figref{Fig:scenario1}のシナリオ1では,ロボットが直線経路を進む途中に歩行者が横断する状況を設定した.\figref{Fig:scenario2}のシナリオ2では,ロボットが狭い通路を走行する際に2名の歩行者がすれ違う状況を設定し,より複雑な環境下でのナビゲーションを想定している.

\begin{figure}[H]
  \centering
  \begin{subfigure}{0.80\textwidth}
    \centering
    \includegraphics[keepaspectratio, scale=0.15]{images/scenario1.pdf}
    \caption{Scenario 1}
    \label{Fig:scenario1}
  \end{subfigure}
  \vspace{10pt}
  \begin{subfigure}{0.80\textwidth}
    \centering
    \includegraphics[keepaspectratio, scale=0.15]{images/scenario2.pdf}
    \caption{Scenario 2}
    \label{Fig:scenario2}
  \end{subfigure}
  \caption{Experiment Scenarios}
  \label{Fig:experiment-scenarios}
\end{figure}

\newpage

評価は以下の4項目に基づいて行った.
\begin{itemize}
  \item 走行時間
  \item 走行距離
  \item 歩行者とロボット間の最小距離
  \item 最小のTime to Collision(TTC)
\end{itemize}
ナビゲーションの効率性は走行時間・距離が小さいほど高く,安全性は最小距離・最小TTCが大きいほど高いと考えられる.
TTCは,移動物体同士が現在の速度と進行方向を維持した場合の衝突するまでの時間を示し,以下の式で計算される.
\setlength{\jot}{1em}
\begin{align}
  \mathbf{v}_{robot} = \begin{bmatrix} v_{robot,x} \\ v_{robot,y} \end{bmatrix}, \quad 
  \mathbf{v}_{ped} = \begin{bmatrix} v_{ped,x} \\ v_{ped,y} \end{bmatrix} \\
  \mathbf{v}_{relative} = \mathbf{v}_{robot} - \mathbf{v}_{ped} \\
  \text{TTC} = \frac{\text{d}}{\|\mathbf{v}_{relative}\|}
\end{align}
ここで,$d$はロボットと歩行者の距離,$\mathbf{v}_{robot}$はロボットの速度ベクトル,$\mathbf{v}_{ped}$は歩行者の速度ベクトルである.

\section{結果と考察}\label{sec:exp-sim-result}
\figref{Fig:scenario1-result},\figref{Fig:scenario2-result}は,予測結果を用いないナビゲーションとロボットの行動を考慮した予測結果を用いるナビゲーションの比較結果である.棒グラフは平均値,エラーバーは標準偏差を表す.
まず,\figref{Fig:scenario1-result}に示すシナリオ1の結果を比較すると,走行時間が約1.8倍,走行距離が約1.01倍と悪化した一方で,最小距離が約13倍,最小TTCが約17倍と大幅に改善した.
次に,\figref{Fig:scenario2-result}に示すシナリオ2の結果では,走行時間が約1.5倍,走行距離が約1.01倍,歩行者2の最小距離が0.9倍と悪化したが,歩行者1の最小距離が1.84倍,歩行者1の最小TTCが約1.3倍,歩行者2の最小TTCが約1.1倍と改善した.
これらの結果から,予測結果を用いることで,ロボットのナビゲーションがより安全になる可能性があることが示唆される.一方で,効率性の面では課題が残ることも確認された.

\begin{figure}[H]
  \centering
 \includegraphics[keepaspectratio, scale=0.53]
      {images/scenario1_result.pdf}
  \caption{Scenario1 result}
 \label{Fig:scenario1-result}
\end{figure} 

\begin{figure}[H]
  \centering
 \includegraphics[keepaspectratio, scale=0.44]
      {images/scenario2_result.pdf}
  \caption{Scenario2 result}
 \label{Fig:scenario2-result}
\end{figure} 

\newpage

シナリオ1では,歩行者との最小距離と最小TTCが大幅に改善されており,予測結果を活用することで早期に安全に停止を行い,歩行者との接触リスクを低減できる一方,回避行動に伴う走行時間・距離の増加が生じたと考えられる.つまり,安全性が向上する一方で,効率性の面では課題が残るような結果となった.

シナリオ2では,歩行者1に対する最小距離と最小TTCが改善されているが,歩行者2との最小距離が悪化している.これは,狭い通路でのすれ違い時にロボットが歩行者2に対してより接近する状況が発生したことを意味している.この結果から,狭い空間でのナビゲーションにおいては,予測結果の精度には改善の余地があると考えられる.また,本研究の学習に用いたデータセットはいずれも屋外の広い空間のデータであり,狭い環境での学習サンプルが不足していたことも一因と考えられる.

本実験で用いたシミュレータの歩行者は,一定の線形軌道を常に一定速度で歩行しており,現実の歩行者の動きとは大きく異なっている.つまり,歩行者同士や歩行者とロボット間など,全ての移動体の間に相互作用が存在しない.その結果,相互作用を重視して学習したモデルと実験環境の性質が乖離し,予測性能が低下した可能性がある.さらに,予測結果を用いる場合の標準偏差が大きい要因として,以下の点が考えられる.
\begin{itemize}
  \item 正規分布からサンプリングされた軌道が実際の軌道と大きく異なる場合がある
  \item 0.4秒ごとに反映される予測結果がロボットの経路計画にチャタリングを引き起こす
  \item 前後の予測結果(例えば,$t-1\text{と}t$)に一貫性がない場合がある
\end{itemize}
これらの課題を解決するためには,狭い空間など多様な環境,相互作用を含むデータセットの活用や,提案したナビゲーションシステムの予測結果の扱い方の改善などが今後の検討課題となる.

\newpage

%

\chapter{結論}
\label{chap:conclusion}
%
%\input{introduction/preface}
%
%!TEX root = ../thesis.tex

\section{まとめ}
本研究では,移動ロボットのための深層学習を用いた歩行者の位置予測と,そのナビゲーションへの応用を検討した.具体的には,ロボットの行動を考慮した歩行者の軌道予測手法を提案し,その有効性をシミュレーションおよび実環境での実験を通じて評価した.

提案手法では,グラフアテンションネットワークを用いて歩行者間およびロボットとの相互作用をモデル化し,将来の歩行者の位置を高精度に予測できることを確認した.シミュレーション実験において,予測結果をナビゲーションに組み込むことで,ロボットの安全性が向上することが確認された.一方で,走行時間や走行距離といった効率性の面では課題が残ることが明らかになった.

実環境での実験では,ロボットの行動を考慮することで,歩行者の軌道予測の精度が向上することが示された.特に,ロボットの動きが歩行者の最終的な位置に大きな影響を与えることを示す結果が得られた.
以上の結果から,提案手法は移動ロボットのナビゲーションにおいて有用である可能性があると考えられるが,さらなる効率性の向上に向けた手法の改良が課題として残る.

\section{今後の展望}

\newpage
%

%ここにディレクトリのパスを追加していく
%
%% Back Matter
\backmatter{}
%
\chapter{提案手法}
\label{chap:suggest}
%
本章では, 従来手法をベースとする提案手法についての概要, 提案手法における学習フェーズ, テストフェーズ, 目標方向, ネットワーク構造についての5節に分けて述べる.
%\input{introduction/preface}
%
%!TEX root = ../thesis.tex

\section{提案手法の概要}
従来手法で用いていたデータセットと学習器の入力へ, 「直進」「左折」などの目標方向を追加する. これにより, 学習器の出力による自律移動において, 経路を選択する機能の追加を行った. なお, 追加した要素以外は従来手法と同様である.

% \subsection{RoboCup}

% \begin{figure}[hbtp]
%   \centering
%  \includegraphics[keepaspectratio, scale=0.4]
%       {images/deeplearning_model.png}
%  \caption{Neural network}
%  \label{Fig:Neural network}
% \end{figure}

% \subsubsection{etc...}
\newpage

%!TEX root = ../thesis.tex

\section{学習フェーズ}
提案手法で用いる学習フェーズのシステムを\figref{Fig:suggest_learning_sys}に示す. 自律移動を行う地図を用いたルールベース制御器から目標方向を生成し, データセットに加えている. 
なお, 厳密にはルールベース制御を構成するwaypoint\_navにより目標方向を生成している. 
提案手法では, 
% \figref{Fig:overview}に示すように
LiDARとオドメトリを入力とする地図を用いたルールベース制御器による自律移動を, カメラ画像と目標方向を用いて模倣学習する.

% \vspace{3cm}
\begin{figure}[hbtp]
     \centering
    \includegraphics[keepaspectratio, scale=0.45]
         {images/conventional.png}
    \caption{Learning phase system of conventional method}
    \label{Fig:conventional}
   \end{figure}   

\begin{figure}[hbtp]
  \centering
 \includegraphics[keepaspectratio, scale=0.45]
      {images/suggest_learning_sys2.png}
 \caption{Learning phase system of proposed method}
 \label{Fig:suggest_learning_sys}
\end{figure}

% \begin{figure}[hbtp]
%   \centering
%  \includegraphics[keepaspectratio, scale=0.6]
%       {images/overview.png}
%  \caption{Overview learning phase}
%  \label{Fig:overview}
% \end{figure}

% \subsubsection{etc...}
\newpage

%!TEX root = ../thesis.tex

\section{テストフェーズ}
提案手法におけるテストフェーズでは\figref{Fig:suggest_test_phase}で示すように, 学習器の入力へ目標方向を加えた. なお, テストフェーズも学習フェーズと同様に, 地図を用いたルールベース制御器から目標方向を生成している. 本来ならば, 目的地までカメラ画像のみで自律移動するためには, 目標方向を画像から自動的に作成する仕組みが必要となる. 
% figに動作の様子を示す. 
カメラ画像と目標方向を入力した学習器の出力による自律走行時, 目標方向によって任意の経路を選択する.

\vspace{3cm}

\begin{figure}[hbtp]
  \centering
 \includegraphics[keepaspectratio, scale=0.45]
      {images/suggest_test_phase.png}
 \caption{Learning phase system of proposed method}
 \label{Fig:suggest_test_phase}
\end{figure}

% \subsubsection{etc...}
\newpage

%!TEX root = ../thesis.tex

\section{目標方向}
本研究で用いた目標方向と, そのデータ形式である目標方向指令について述べる. 目標方向を\figref{Fig:direction}に示す. 経路と分岐路において「道なり」に走行(Go straight), 分岐路において「直進(Go straight)」, 「左折(Turn left)」, 「右折(Turn right)」の3つとする.

\begin{figure}[hbtp]
  \centering
 \includegraphics[keepaspectratio, scale=0.45]
      {images/direction.png}
 \caption{Target direction}
 \label{Fig:direction}
\end{figure}

学習器には, 上記の3つの目標方向を要素数3, 次元数1のint型の配列で表現した”目標方向指令”を入力する. 目標方向指令のデータ形式を\tabref{table:direction}に示す.

\begin{table}[hbtp]
  \caption{Target direction list}
  \label{table:direction}
  \centering
  \begin{tabular}{|c|c|c|c|}
    \hline
    Target Direction  & Go straight & Turn left & Trun right\\
    \hline
    Data & [100, 0, 0] & [0, 100, 0] & [0, 0, 100]\\
    \hline
  \end{tabular}
\end{table}

% \begin{figure}[hbtp]
%   \centering
%  \includegraphics[keepaspectratio, scale=0.45]
%       {images/direction.png}
%  \caption{Learning phase system of proposed method}
%  \label{Fig:direction}
% \end{figure}

% \subsubsection{etc...}
\newpage

%!TEX root = ../thesis.tex

\section{ネットワーク構造}
提案手法で用いた学習器のネットワークを\figref{Fig:network_structure}に示す. また, ハイパーパラメータについて\tabref{table:param1}に示す. 64x48のRGB画像を入力とする入力層1層, 畳み込み層3層, 全結合層2層を持つ6層のCNNと, このCNNの出力と目標方向指令を入力する入力層1層, 全結合層2層, 出力層1層の全10層から構成されている. 出力はヨー方向の角速度である.

\begin{figure}[hbtp]
  \centering
 \includegraphics[keepaspectratio, scale=0.43]
      {images/network_structure2.png}
 \caption{Structure of network}
 \label{Fig:network_structure}
\end{figure}

\begin{table}[hbtp]
  \caption{Parameters of network}
  \label{table:param1}
  \centering
  \begin{tabular}{|c|c|}
    \hline
    Input data & Image(64x48 pixels, RGB channels), Target direction \\
    \hline
    Optimizer & Adam($alpha = 0.001, beta1 = 0.9, beta2 =  0.999, eps = 1e^{-2}$)\\
    \hline
    Loss function & Softmax-cross-entropy\\
    \hline
    Output data & Angular velocity\\
    \hline
  \end{tabular}
\end{table}

% \begin{figure}[hbtp]
%   \centering
%  \includegraphics[keepaspectratio, scale=0.45]
%       {images/network_structure.png}
%  \caption{Learning phase system of proposed method}
%  \label{Fig:network_structure}
% \end{figure}

% \subsubsection{etc...}
\newpage

%

%

\end{document}
