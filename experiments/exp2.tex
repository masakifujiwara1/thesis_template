%!TEX root = ../thesis.tex

\section{課題点と2つのアプローチによる実験}
我々が行ってきた研究では, 簡易的なシミュレータ上で提案手法が有効だと確認されている. そのため, 次の段階として実環境における提案手法の有効性を検証することを試みた. そこで, 新たに顕在化した課題点は以下の2つの点である.

\begin{itemize}
  \item 実験条件(主に光である)を揃える関係上, 実験を行う時間帯を光の変化が少ない夜間に固定する必要があるため, 1日に実験を行える時間が少なく, 1回の学習に何日も費やす必要がある
  \item 長時間の学習に耐えられるだけのバッテリ容量がロボットにない
\end{itemize}

これらの課題点から, 学習時間の短縮が必要であると判断した. そのため, 2つのアプローチを提案し, 学習量を削減する.
\par
この節では, まず, 従来の実験を簡単に紹介する. 次に, 2つのアプローチについての詳細と行った実験を述べる. 最後に, アプローチを試みる前と各アプローチによる実験結果を比較し, 議論を行う.

\subsection{従来の実験}

\begin{itemize}
  \item 実験環境

  \begin{figure}[hbtp]
  \centering
 \includegraphics[keepaspectratio, scale=0.35]
      {images/tsudanuma2-3_sim.png}
 \caption{Experimental environment from \cite{mech}}
 \label{Fig:tsudanuma2-3_sim}
\end{figure}

\begin{figure}[hbtp]
  \centering
 \includegraphics[keepaspectratio, scale=0.35]
      {images/select_patarn.png}
 \caption{Selecting a path at the T-junction from \cite{mech}}
 \label{Fig:select_patarn}
\end{figure}

  \item 実験方法
  \item 実験結果
\end{itemize}

% \begin{figure}[hbtp]
%   \centering
%  \includegraphics[keepaspectratio, scale=0.8]
%       {images/RaspberryPiMouse.png}
%  \caption{Example}
%  \label{Fig:Example}
% \end{figure}

% \newpage
