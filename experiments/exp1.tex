%!TEX root = ../thesis.tex

\section{実験要件}
実験には下記のコンピュータとソフトウェアを用いた. 

\begin{enumerate}
  \item コンピュータ\\
  OS: Ubuntu 20.04 LTS\\
  ROS: Noetic\\
  CPU: intel Core i7-10700F(4.8GHz/8コア/16スレッド)\\
  DRAM: 32GB DDR4(3200/8GB×4)
  \item nav\_cloning(学習器, 統合環境)\\
  \url{https://github.com/open-rdc/nav_cloning}
  \item waypoint\_nav(移動目標地点, 目標方向を出力)\\
  \url{https://github.com/open-rdc/waypoint_nav}
  \newpage
  \item turtlebot3 関連\\
  \url{https://github.com/open-rdc/turtlebot3}
  \item navigation(ナビゲーションパッケージ)\\
  \url{https://github.com/ros-planning/navigation}
\end{enumerate}

\subsection{実験装置(シミュレータ)}

\begin{itemize}
  \item ロボット

ロボットモデルは前報\cite{okada1}\cite{okada2}と同様, \figref{Fig:waffle_pi}に示すように, TurtleBot3 Waffle\_piへ3つのカメラを追加したモデルを用いる.

\begin{figure}[hbtp]
  \centering
 \includegraphics[keepaspectratio, scale=0.22]
      {images/Waffle_pi.png}
 \caption{TurtleBot3 waffle\_pi with 3 cameras}
 \label{Fig:waffle_pi}
\end{figure}

\item 環境\\
シミュレータ環境として, オープンソースの3DロボットシミュレータGazeboを用いる. \figref{Fig:sim}に示すようなGazebo上で千葉工業大学2号館3階を模した実験環境を対象に実験を行う.

\begin{figure}[hbtp]
  \centering
 \includegraphics[keepaspectratio, scale=0.12]
      {images/tsudanuma2-3_simorg.png}
 \caption{Experimental enviroment of simulator}
 \label{Fig:sim}
\end{figure}

\end{itemize}

\subsection{実験方法}

\begin{figure}[hbtp]
  \centering
 \includegraphics[keepaspectratio, scale=0.5]
      {images/sim_explain.png}
 \caption{Experimental enviroment of simulator}
 \label{Fig:sim_explain}
\end{figure}

\begin{figure}[hbtp]
  \centering
 \includegraphics[keepaspectratio, scale=0.15]
      {images/select.png}
 \caption{Experimental enviroment of simulator}
 \label{Fig:select}
\end{figure}

\newpage
