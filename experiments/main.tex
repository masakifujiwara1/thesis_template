\chapter{学習時間の短縮化の提案}
\label{chap:experiments}
% この章では, 1節で我々が行ってきた研究\cite{mech}の実験(以下, 「従来の実験」と称する)を実環境に移行する際に, 新たに顕在化した課題点について述べる. また, 課題を解決するための2つのアプローチを提案し, 実験と検証を行う. 2節では, 実験に簡易的なシミュレータを用いる問題点を述べ, 解決策を提示する. 3節では, 実環境で実験を行い, 実環境における提案手法の有効性を検証する.  
この章では, 1節で, 前章で問題となった学習時間の長さについて議論する. 2節では, 前章の実験を基に, 学習ステップ数を減らした実験を行う. 3, 4節では, 問題を解決するための2つのアプローチをそれぞれ提案し, 実験と検証を行う. 5節では, 実験に簡易的なシミュレータを用いる問題点を述べ, 解決策を提示する. 6節では, 実環境で実験を行い, 実環境における提案手法の有効性を検証する.  
%
%\input{experiments/preface}
%
% %!TEX root = ../thesis.tex

\section{実験要件}
実験には下記のコンピュータとソフトウェアを用いた. ロボットモデルは前報\cite{okada1}\cite{okada2}と同様, \figref{Fig:waffle_pi}に示すように, TurtleBot3 Waffle\_piへ3つのカメラを追加したモデルを用いる.

\begin{figure}[hbtp]
  \centering
 \includegraphics[keepaspectratio, scale=0.3]
      {images/Waffle_pi.png}
 \caption{TurtleBot3 waffle\_pi with 3 cameras}
 \label{Fig:waffle_pi}
\end{figure}

\begin{enumerate}
  \item コンピュータ\\
  OS: Ubuntu 20.04 LTS\\
  ROS: Noetic\\
  CPU: intel Core i7-10700F(4.8GHz/8コア/16スレッド)\\
  DRAM: 32GB DDR4(3200/8GB×4)
  \item nav\_cloning(学習器, 統合環境)\\
  \url{https://github.com/open-rdc/nav_cloning}
  \item waypoint\_nav(移動目標地点, 目標方向を出力)\\
  \url{https://github.com/open-rdc/waypoint_nav}
  \item turtlebot3 関連\\
  \url{https://github.com/open-rdc/turtlebot3}
  \item navigation(ナビゲーションパッケージ)\\
  \url{https://github.com/ros-planning/navigation}
\end{enumerate}
% \newpage

%!TEX root = ../thesis.tex

\section{課題点と2つのアプローチによる実験}
我々が行ってきた研究では, 簡易的なシミュレータ上で提案手法が有効だと確認されている. そのため, 次の段階として実環境における提案手法の有効性を検証することを試みた. そこで, 新たに顕在化した課題点は以下の2つの点である.

\begin{itemize}
  \item 実験条件(主に光である)を揃える関係上, 実験を行う時間帯を光の変化が少ない夜間に固定する必要があるため, 1日に実験を行える時間が少なく, 1回の学習に何日も費やす必要がある
  \item 長時間の学習に耐えられるだけのバッテリ容量がロボットにない
\end{itemize}

これらの課題点から, 学習時間の短縮が必要であると判断した. そのため, 2つのアプローチを提案し, 学習量を削減する.
\par
この節では, まず, 従来の実験を簡単に紹介する. 次に, 2つのアプローチについての詳細と行った実験を述べる. 最後に, アプローチを試みる前と各アプローチによる実験結果を比較し, 議論を行う.

\subsection{従来の実験}

\begin{itemize}
  \item 実験環境

  \begin{figure}[hbtp]
  \centering
 \includegraphics[keepaspectratio, scale=0.35]
      {images/tsudanuma2-3_sim.png}
 \caption{Experimental environment from \cite{mech}}
 \label{Fig:tsudanuma2-3_sim}
\end{figure}

\begin{figure}[hbtp]
  \centering
 \includegraphics[keepaspectratio, scale=0.35]
      {images/select_patarn.png}
 \caption{Selecting a path at the T-junction from \cite{mech}}
 \label{Fig:select_patarn}
\end{figure}

  \item 実験方法
  \item 実験結果
\end{itemize}

% \begin{figure}[hbtp]
%   \centering
%  \includegraphics[keepaspectratio, scale=0.8]
%       {images/RaspberryPiMouse.png}
%  \caption{Example}
%  \label{Fig:Example}
% \end{figure}

% \newpage

%!TEX root = ../thesis.tex

\section{実環境に似たシミュレータ環境による実験}
これまでの実験では, 簡易的なシミュレータ環境を用いてきた. しかし, 実験に簡易的なシミュレータ環境を用いるには問題があり, 以下の点である.

\begin{itemize}
  % \item 屋内を模している環境であるため, 影は少ないはずだが, \figref{Fig:sim_shadow}に示すように全体的に影が写り込んでしまっている

  % \begin{figure}[hbtp]
  %   \centering
  %  \includegraphics[keepaspectratio, scale=0.37]
  %       {images/sim_up.png}
  %  \caption{Ratio of data by distance from target path in test phase}
  %  \label{Fig:sim_up}
  % \end{figure} 

  % \begin{figure}[h]
  %   \centering
  %   \begin{minipage}[b]{67mm}
  %     \centering
  %     \includegraphics[width=67mm, height=50.5mm]{images/sim_robot.png}
  %     \caption*{(a)}
  %   \end{minipage} 
  %   % \newpage
  %   % \hspace{0.03\columnwidth}
  %   \begin{minipage}[b]{67mm}
  %     \centering
  %     \includegraphics[width=67mm, height=50.5mm]{images/sim_up.png}
  %     \caption*{(b)}
  %   \end{minipage}
  %   \caption{Number of data per command per 10000steps in conventional experiments}
  %   \label{Fig:sim_shadow}
  % \end{figure}

  \item \figref{Fig:sim_shadow} (a)に示すように, 環境の大半が灰色や白のみで構成されているため, \figref{Fig:sim_shadow} (b)のように視覚による特徴が乏しい. 
  
   \begin{figure}[h]
    \centering
    \begin{minipage}[b]{67mm}
      \centering
      \includegraphics[width=50mm, height=36mm]{images/sim_up.png}
      \caption*{(a) A bird's eye view of the robot}
    \end{minipage} 
    % \newpage
    % \hspace{0.03\columnwidth}
    \begin{minipage}[b]{67mm}
      \centering
      \includegraphics[width=50mm, height=36mm]{images/sim_robot.png}
      \caption*{(b) Robot Perspective}
    \end{minipage}
    \caption{Simple simulator environment}
    \label{Fig:sim_shadow}
  \end{figure}

\end{itemize}

この問題は, 学習フェーズにおいて取得するカメラ画像で学習器を訓練する際に, 大きな影響を及ぼす可能性がある. そのため, ロボットの視覚であるカメラ画像で, より多くの視覚的特徴をとらえるために, \figref{Fig:real_sim}に示すように実環境に似たシミュレータ環境を作成した. 

\begin{figure}[h]
  \centering
  \begin{minipage}[b]{67mm}
    \centering
    \includegraphics[width=50mm, height=36mm]{images/real_sim_up.png}
    \caption*{(a) A bird's eye view of the robot}
  \end{minipage} 
  % \newpage
  % \hspace{0.03\columnwidth}
  \begin{minipage}[b]{67mm}
    \centering
    \includegraphics[width=50mm, height=36mm]{images/real_sim_robot.png}
    \caption*{(b) Robot Perspective}
  \end{minipage}
  \caption{Simulator environment similar to real environment}
  \label{Fig:real_sim}
\end{figure}

% \begin{figure}[hbtp]
%   \centering
%  \includegraphics[keepaspectratio, scale=0.8]
%       {images/RaspberryPiMouse.png}
%  \caption{Example}
%  \label{Fig:Example}
% \end{figure}

\newpage

%!TEX root = ../thesis.tex

\section{実環境の実験}
これまでの節では, いずれもシミュレータ上での実験を行ってきたが, 実験を実環境に移す. それから, 実環境における提案手法の有効性を検証する.

\subsection{実験装置(実環境)}
\begin{itemize}
  \item ロボット
  
  ロボットは前報\cite{okada1}と同様, \figref{Fig:gamma}に示すように, 3つのカメラを搭載したロボットを用いる.

  \vspace{2cm}
  
  \begin{figure}[hbtp]
    \centering
   \includegraphics[keepaspectratio, scale=0.7]
        {images/gamma2.png}
   \caption{Experimental setup from \cite{okada1}}
   \label{Fig:gamma}
  \end{figure}

  \newpage

  \item 環境

  \figref{Fig:real_environment}に示すような千葉工業大学津田沼キャンパス2号館3階で実験を行う.

  \begin{figure}[h]
    \centering
    \begin{minipage}[b]{120mm}
      \centering
      \includegraphics[width=40mm]{images/real.png}
      \caption*{(a) One place in the real environment}
    \end{minipage} 
    % \newpage
    % \hspace{0.03\columnwidth}
    \begin{minipage}[b]{120mm}
      \centering
      \includegraphics[width=95mm]{images/tsudanuma_structure.png}
      \caption*{(b) structure}
    \end{minipage}
    \caption{Real environment}
    \label{Fig:real_environment}
  \end{figure}
\end{itemize}

\subsection{実験方法}
4.2の実験により, 20000stepほど学習させることで, 十分な成功率が得られる可能性が高いことがわかっている. しかし, 実環境で行う実験時間を削減するため, 実環境に似たシミュレータ上で訓練した学習器をファインチューニングする. なお, 本論文ではファインチューニングによる実験結果の影響を議論しない. 実環境における実験の流れを以下に示す.
\begin{enumerate}
  \item 事前に4.3の実験に倣って, シミュレータ上で10000step学習させる. 
  \item 前段階の学習器を用いて初期値を設定し, 実環境で4.1.2 で示した経路を繰り返し走行させる.
  \item 学習を10000step実行後, テストフェーズに移行する. テストフェーズで正しい順序で経路を選択し, 走行を行えるか確認する.
\end{enumerate}
この一連の流れを 10 回繰り返し行う.

\subsection{実験結果}
実験結果を \figref{Fig:real_result} に示す. この図は, それぞれの走行パターンにおいて正しく経路を選し, 走行できた回数を表している. \tabref{table:real} に実験ごとに全パターンを合計した結果を示す. \tabref{table:real} に示すように, 目標方向に従って 78/120 回, 正しい経路を選択する様子が見られた.

\begin{figure}[hbtp]
  \centering
 \includegraphics[keepaspectratio, scale=0.5]
      {images/real_result.png}
 \caption{Experimental setup from \cite{okada1}}
 \label{Fig:real_result}
\end{figure}

\begin{table}[hbtp]
  \caption{Experimental results}
  \label{table:real}
  \centering
  \begin{tabular}{|c|c|c|}
    \hline
    Experiments & Step & Total results\\
    \hline
    Approach1+2 & 20000 & /120(\%)\\
    \hline
    Real & 20000 & 78/120(65\%)\\
    \hline
  \end{tabular}
\end{table}

\newpage

%
