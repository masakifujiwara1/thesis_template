%!TEX root = ../thesis.tex

\section{Convolution Neural Network}
畳み込みニューラルネットワーク(convolutional neural network:CNN)は
人工ニューラルネットワークのモデルの一種である. このモデルは, 画像や音声などの
多次元の配列で表される複雑なデータを処理するために特別に設計されている.
CNNは次のような特徴を持つ層で構成されている.
\begin{enumerate}
  \item 畳み込み層\\入力データをフィルタ(カーネル)を用いて特徴を抽出する.
  \item プーリング層\\特徴を残しつつ, 畳み込み層の出力を圧縮する. これにより, 画像で
  あればピクセル数が減少し, 計算量が大幅に減らすことができる.
  \item 全結合層\\畳み込み層とプーリング層の出力をまとめて処理する.
\end{enumerate}

% \vspace{5cm}

% \begin{figure}[hbtp]
%   \centering
%  \includegraphics[keepaspectratio, scale=0.7]
%       {images/end-to-end.png}
%  \caption{Structure of end-to-end learning}
%  \label{Fig:end-to-end}
% \end{figure}

% \subsubsection{etc...}
\newpage
