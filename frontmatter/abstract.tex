%!TEX root = ../thesis.tex
\chapter*{概要}
\thispagestyle{empty}
%
\begin{center}
  \scalebox{1.5}{移動ロボットのための深層学習を用いた}\\
  \scalebox{1.5}{歩行者の位置予測とナビゲーションへの応用}\\
\end{center}
\vspace{1.0zh}
%

歩行者と移動ロボットが混在する環境で,ロボットが人間の移動をある程度正確に予測し,安全な経路を動的に計画することは,安全性と効率性の両面から重要である.しかし,従来の研究の多くは主に歩行者同士の相互作用を考慮しており,移動ロボットの動きや,それが歩行者の行動に与える影響を包括的に取り入れた手法は限られていた.本研究では,この課題を解決するために,歩行者同士だけでなくロボットとの相互作用を同時にモデル化する軌道予測モデルを提案する.

本モデルを各種データセットで学習し,複数の評価指標でベースラインモデルと比較した結果,約9%の精度向上を達成した.また,実環境の実験では,ロボットの行動を考慮した予測が,考慮しない場合に比べて歩行者の軌道をより高精度に捉えることを確認している.さらに,予測結果をナビゲーションシステムに応用することで,動的な環境下でも安全性を高められることを実証した.一方,効率的な経路計画の面では今後さらなる検討が必要である.

本研究は,歩行者とロボットの双方向的な相互作用を反映した軌道予測モデルを開発・検証し,その有効性を示した.
具体的には,グラフアテンションネットワークを用いて,歩行者とロボットの相互作用を同時にモデル化し,軌道予測の精度を向上させることを目指した.さらに,提案モデルを用いて,動的な環境下でのロボットのナビゲーションシステムの安全性を高めることを確認した.

\vspace{1.0zh}
\begin{flushleft}
キーワード: 自律移動ロボット,Navigation,ROS,深層学習
\end{flushleft}
%
\newpage
%%
\chapter*{abstract}
\thispagestyle{empty}
%
\begin{center}
  \scalebox{1.3}{title}
\end{center}
\vspace{1.0zh}
%


keywords:
