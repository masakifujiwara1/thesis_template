%!TEX root = ../thesis.tex
\chapter*{概要}
\thispagestyle{empty}
%
\begin{center}
  \scalebox{1.5}{移動ロボットのための深層学習を用いた}\\
  \scalebox{1.5}{歩行者の位置予測とナビゲーションへの応用}\\
\end{center}
\vspace{1.0zh}
%

歩行者と移動ロボットが混在する環境では,ロボットが歩行者の移動を正確に予測し,歩行者を避けるように経路を動的に計画することは,安全性と効率性の両面から重要である.しかし,従来の研究の多くは主に歩行者同士の相互作用を考慮しており,移動ロボットの動作や,それが歩行者の行動に与える影響を包括的に取り入れた手法は限られていた.本研究では,この課題を解決するために,歩行者同士だけでなくロボットとの相互作用を同時にモデル化する軌道予測モデルを提案する.

提案モデルでは,グラフアテンションネットワーク(Graph Attention Network, GAT)を用いて,歩行者とロボットの相互作用を動的に捉える.これにより,従来の手法と比較して軌道予測の精度を向上させることを目指した.本モデルを既存の複数のデータセットで学習させた結果,ベースラインモデルと比較して約9%の精度向上を達成した.また,実環境の実験では,ロボットの行動を考慮することで歩行者の軌道予測精度が向上することを確認した.

さらに,本研究では提案モデルの予測結果を利用したナビゲーションシステムを開発し,シミュレータ環境でその性能を検証した.加えて,歩行者とロボット間の接触リスクを複数の指標で定量化し,その低減効果を評価した結果,本システムを導入することにより,予測結果を用いない場合と比べて接触リスクが低下することを確認した.

% \vspace{1.0zh}
\begin{flushleft}
キーワード: 深層学習,自律移動ロボット,ナビゲーション
\end{flushleft}
%
\newpage
%%
% \vspace{-40pt}
\chapter*{abstract}
\thispagestyle{empty}
%
\vspace{-30pt}
\begin{center}
  \scalebox{1.2}{Pedestrian Position Prediction Using Deep Learning for Mobile Robot}\\
  \scalebox{1.2}{and Its Application to Navigation}
\end{center}
\vspace{1.0zh}
%
In environments where pedestrians and mobile robots coexist, it is crucial for robots to accurately predict pedestrian movements and dynamically plan their paths to avoid pedestrians, ensuring both safety and efficiency. However, many traditional studies have primarily considered interactions among pedestrians, with limited methods comprehensively incorporating the robot's actions and their influence on pedestrian behavior. To address this challenge, this study proposes a trajectory prediction model that simultaneously models interactions between pedestrians and robots.

The proposed model utilizes Graph Attention Networks (GAT) to dynamically capture the interactions between pedestrians and robots. This approach aims to improve trajectory prediction accuracy compared to conventional methods. Training the model on multiple existing datasets resulted in approximately a 9\% increase in accuracy over baseline models. Additionally, experiments in real-world environments confirmed that accounting for robot behavior enhances the accuracy of pedestrian trajectory predictions.

Furthermore, this research develops a navigation system utilizing the proposed model's prediction results and validates its performance in a simulator environment. Contact risks between pedestrians and robots are quantified using multiple indicators, and the effectiveness of risk reduction is evaluated. The results demonstrate that introducing this system reduces contact risks compared to scenarios without utilizing prediction results.

% \vspace{-13pt}
\begin{flushleft}
keywords: deep learning, autonomous mobile robots, navigation
\end{flushleft}