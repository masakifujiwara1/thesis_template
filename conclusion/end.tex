%!TEX root = ../thesis.tex

\section{まとめ}
本研究では,移動ロボットのための深層学習を用いた歩行者の位置予測とそのナビゲーションへの応用について検討した.具体的には,ロボットの行動を考慮した歩行者の軌道予測手法を提案し,その有効性をシミュレーションおよび実環境での実験を通じて評価した.

提案手法では,グラフアテンションネットワークを用いて歩行者間およびロボットとの相互作用をモデル化し,将来の歩行者の位置を高精度に予測することができた.シミュレーション実験では,予測結果をナビゲーションに応用することで,ロボットの安全性が向上することが確認された.一方で,効率性の面では課題が残ることも明らかになった.

実環境での実験では,ロボットの行動を考慮することで,歩行者の軌道予測の精度が向上することが示された.特に,ロボットの動きが歩行者の最終的な位置に大きな影響を与えることが確認された.

以上の結果から,提案手法は移動ロボットのナビゲーションにおいて有用である可能性があるが,さらなる効率性の向上が求められる.

\section{今後の展望}

\newpage