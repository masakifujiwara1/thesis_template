%!TEX root = ../thesis.tex

\section{まとめ}
本研究では,移動ロボットのための深層学習を用いた歩行者の位置予測と,そのナビゲーションへの応用を検討した.具体的には,ロボットの行動を考慮した歩行者の軌道予測手法を提案し,その有効性をシミュレータおよび実環境での実験を通じて評価した.

提案手法では,グラフアテンションネットワークを用いて歩行者間およびロボットとの相互作用をモデル化し,将来の歩行者の位置を高精度に予測できることを確認した.シミュレータ実験においては,予測結果をナビゲーションに統合することで,歩行者とロボット間の接触リスクが低減されることが確認された.一方で,走行時間や走行距離といった効率性の面では課題が残ることが明らかになった.

実環境での実験では,ロボットの行動を考慮することで歩行者の軌道予測の精度が向上することが確認された.特に,ロボットの動きが歩行者の最終的な位置に大きな影響を及ぼすことを示す結果が得られた.
以上の結果から,提案モデルによる予測結果を用いたナビゲーションシステムは,移動ロボットの運用において有用な可能性がある.

\newpage

\section{今後の展望}
本研究で提案した歩行者の軌道予測モデルとナビゲーション手法は,ロボットの行動を考慮した高精度な位置予測が可能であることを示した.一方で,さらなる効率性向上や,リアルタイム性・拡張性の面でいくつかの課題が残されている.今後の展望としては,以下の点が挙げられる.

\begin{itemize}
  \item \textbf{多様な環境条件への適応} \\
  空間が限られた領域や群集密度の高い環境など,より複雑なシナリオに対する予測精度や安全性の検証を進める.
  \item \textbf{相互作用モデルの拡張} \\
  ロボットと歩行者以外にも障害物や複数のロボットを含む複雑な相互作用をモデル化し,さらに汎用的な予測手法への拡張を図る.
  \item \textbf{ナビゲーションシステムの改善} \\
  \ref{sec:exp-sim-result}節で述べた通り,現状のシステムでは個々の歩行者を安定してクラスタリングすることや,予測結果を動作計画と矛盾が生じないように扱うことが困難である.今後は,これらの課題の改善に取り組む予定である.
\end{itemize}

\newpage