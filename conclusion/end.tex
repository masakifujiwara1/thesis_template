%!TEX root = ../thesis.tex

\section{まとめ}
本研究では,移動ロボットのための深層学習を用いた歩行者の位置予測と,そのナビゲーションへの応用を検討した.具体的には,ロボットの行動を考慮した歩行者の軌道予測手法を提案し,その有効性をシミュレーションおよび実環境での実験を通じて評価した.

提案手法では,グラフアテンションネットワークを用いて歩行者間およびロボットとの相互作用をモデル化し,将来の歩行者の位置を高精度に予測できることを確認した.シミュレーション実験において,予測結果をナビゲーションに組み込むことで,ロボットの安全性が向上することが確認された.一方で,走行時間や走行距離といった効率性の面では課題が残ることが明らかになった.

実環境での実験では,ロボットの行動を考慮することで,歩行者の軌道予測の精度が向上することが示された.特に,ロボットの動きが歩行者の最終的な位置に大きな影響を与えることを示す結果が得られた.
以上の結果から,提案手法は移動ロボットのナビゲーションにおいて有用である可能性があると考えられるが,さらなる効率性の向上に向けた手法の改良が課題として残る.

\section{今後の展望}

\newpage